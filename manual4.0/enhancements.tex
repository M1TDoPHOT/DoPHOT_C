

\centerline {\bf 11. VERSION 2.0 and 3.0 ENHANCEMENTS}

Since its first release (version 1.0), several enhancements have been 
incorporated in succeeding versions 2.0 and 3.0. These are in addition to 
bug fixes and other minor changes that have improved the running of DoPHOT.
We have thus far been able to keep DoPHOT reverse compatible: specifically 
DoPHOT should run exactly the same way (aside from bug fixes) with your 
modified parameters file, with the only change being to point to the new
DEFAULT parameters file and, as of 4.0, a slight change in how empty 
parameters should be designated in the modified parameter files.  
Details in the Appendix.

\centerline {\bf 11.1 Fixed Position Warmstarts}

An option introduced with version 2.0, called a ``fixed
position warmstart'' permits a user to specify the $x$ and $y$
coordinates of stars.  The photometry reported for
appropriately flagged stars is then the result of a two
parameter fit (sky and central intensity), with the two
position parameters fixed at the specified values.
This option might be used to advantage if one had a deep
frame taken in good seeing, and wanted to do photometry (or
set upper limits) on less deep images obtained at a
different epoch, or an image obtained with a different
filter.

While we had long considered this option among those we had
hoped to implement, we were spurred to do so by the success
of D. Bennett of the Lawrence Livermore Laboratory's MACHO
project.  At the 178th meeting of the American Astronomical
Society in Seattle,  Bennett reported that his implementation of the
fixed position idea in a program called SoDoPHOT produced
substantial improvement in accuracy over that obtained
using 4 parmeter fits on the less deep frames.  Our
implementation is within {\tt DoPHOT} itself, allowing for new
stars (and galaxies) to be found in addition to those
specified on the input list.  Since the positions for the newly found
objects are not specified, the photometry is based on four parameter fits
in IMPROVE.

{\tt DoPHOT} uses the image type associated with each star to
distinguish between stars for which 4 parameter fits are to
be carried out and those for which the positions are fixed.
In the latter case, the image types in the input file are
augmented by 10.  Hence an object of type 11 is a star for
which 7 parameter fits are to be carried out (to determine
the shape parameters), but for which the magnitude is to be
determined from a 2 parameter fit.  An object of type 17 is
too faint for 7 parameter fits.  The meanings of all image types are
summarized in Appendix C.

The input list is much like the input to an ordinary
``warmstart'', but with the image types augmented by 10.
Since the deregistration between images is likely to be more
than 1\% width of a typical stellar image, the user will need
to transform the positions from his starting list
(henceforth the ``template'') to those of the ``program''
image.  For this purpose one needs to obtain positions of
the brighter stars on the program image.  A working example
of a program to do this ({\tt OFFSET}) is given in the subdirectory
``transform'', described in more detail below.
It uses output from a standard {\tt DoPHOT} run on
the program image in conjunction with the template positions
to produce a warmstart file.  In addition to transforming
the coordinates from the template system, a crude
photometric transformation is also made.  
Preliminary tests indicate that {\tt DoPHOT} does run faster and provides
somewhat better photometry reproducibility when the coordinates
are fixed.  Reports of additional tests by {\tt DoPHOT} users would be 
most welcome.

When the user requests a {\tt DoPHOT} run with fixed positions, the program
goes through the first threshold differently than if the positions
were not fixed (see the crypto program in Section 3 for reference).  
Specifically, rather than run {\tt ISEARCH}, {\tt DoPHOT} goes directly to 
{\tt IMPROVE}, calculating only two-parameter fits for the appropriate objects
read in from a template object list.  It is a good idea at this point
to have at least rough estimates of the offsets between
the template and program photometry results.  Nevertheless, if these
estimates are poor, it is the job of this first pass in {\tt IMPROVE} to
improve the sky values and central intensities of
each star using some initial guess of the shape parameters.  
Then, {\tt DoPHOT}
re-enters the ``standard'' reduction stream in {\tt SHAPE} where full
seven-parameter fits are performed on stars will sufficient S/N.  
Additional passes through {\tt DoPHOT} allow the program to find stars that
-- for whatever reason -- may not be in the template list.  The only 
difference between these subsequent passes and ``normal'' passes in
{\tt DoPHOT} is that the positions of most objects from the template
list remain fixed and only two-parameter fits (for sky and central 
intensity) are performed in {\tt IMPROVE}.

\centerline{\bf 11.2 Running Fixed-Position Warmstarts}

Fixed-position warmstarts are as yet untested with version 4.0.  
We are happy to receive reports.

Within the {\bf verif\_data} directory are a number of files designed
to allow the user to test the fixed-position option in {\tt DoPHOT}.  This 
section provides step-by-step instructions of how to run these tests and
make sure that {\tt DoPHOT} is running correctly on your machine.  For 
simplicity, the directories and file names will be referred to in their
!TEX encoding = UTF-8 UnicodeUnix versions;  it should be obvious how to translate these to 
the corresponding VMS names.
The basic aims of the test described below are (1) reduce two frames of
the same field obtained on different nights and exhibiting large 
positional offsets, (2) make a ``template'' file that will serve as the
input to the fixed-position version of {\tt DoPHOT}, (3) reduce a frame using
this template file.  {\it Be sure to copy all files from the} {\bf
verif\_data} {\it directory to the} {\bf working\_data} {\it or else you will
overwrite the files supplied with {\tt DoPHOT}.}

Two disk-fits files are supplied in the {\bf verif\_data} directory called
{\bf image\_in\_template} and {\bf image\_in\_program}.  Assuming you have 
now compiled {\tt DoPHOT} 4.0 (see Section 4), simply run the program twice to
reduce these frames.  The input parameter file for the first image is
(naturally) {\bf param\_in\_template} and for the second image 
{\bf param\_in\_program}.   The results of these runs will be two files
({\bf obj\_out\_template1} and {\bf obj\_out\_program1}, respectively) that
correspond to full-blown, standard {\tt DoPHOT} reductions of these frames.
You should check to ensure that these files are identical to their 
counterparts in the {\bf verif\_data} directory as described at the end of
Section 4.1.   You might also want to look at the various output (subtracted)
images generated by these (and subsequent) {\tt DoPHOT} runs.
These reductions will serve as benchmarks for the steps
that follow.  Note that if all went well, 2382 stars were found in the
template image, and only 1838 in the program image.  Both images are of the
same field (NGC~2031 in the LMC), but the seeing is somewhat worse on the
program image.

The whole point of the fixed-position option is that one can, and should
use positions obtained from a deeper or better-seeing frame to reduce 
any given image.  Although we fully reduced the program image above, we
really didn't have to.  All we need are enough positions to be able to
precisely transform the template coordinates to the program positions.  
In general, only a few dozen stars are needed to do this, not 1838.  Thus,
a more efficient method of running the fixed-position option is to reduce
the program frame ``once over lightly''.  To illustrate this, run 
{\tt DoPHOT} 4.0 using {\bf param\_program1} as the input parameter file.  If you
look at this parameter file (and study the meanings of the relevant 
parameters in Appendix A) you will see that this reduction of the program
image stops at a very high minimum threshold (200, in fact); 
no attempt is made to find
and measure all the stars.  In fact, if all went well, only 315 stars
will have been found, and you should note that the run was much faster then
when you reduced the frame to completion.  The resulting file, 
{\bf obj\_out\_program1} will only be used to transform the template results
to the program frame coordinate system.

To do this, we have supplied a routine called {\tt OFFSET} in the 
{\bf transform} subdirectory.  {\it This routine is not thoroughly tested
and is intended only to give the user an idea of how one might change
the program coordinates to the template system.}  That said, one should now
run {\tt OFFSET}, answering the questions as follows.
The primary file name is {\bf obj\_out\_program1};  this file represents the
``target'' coordinate system.  The secondary file name is 
{\bf obj\_out\_template1}.  The ratio of the scales is 1.0 and a tolerance
of 0.02 (in pixels) is fine.   The resulting offsets in $x$ and $y$ are -1.45 and
9.17 pixels, respectively;  the scales are nearly 1.0, and the rotations
tiny.  After this information is spewed out, answer the next
query with a `yes' and name the 
output file {\bf obj\_in\_template2}.  You should answer the final question
with `no'.   
For the curious, {\tt OFFSET} matches coordinate lists 
by identifying similar triangles.  It is supposed to be able to handle 
linear offsets (assuming the $x$ and $y$ scales are the same), rotations,
and reflections.   You might want to supply a different routine to do this
task;  however, {\tt OFFSET} is still useful as a guide for what the 
format of the output file should be.

Now that you have a transformed version of the template file, run 
{\tt DoPHOT} 4.0 again using {\bf param\_program2} as the input parameter
file.  The key feature of this parameter file is that the parameter
{\tt FIXPOS} is set equal to {\tt YES}, and the {\tt OBJECT\_IN}
parameter is set equal to the file {\bf obj\_in\_template2}.  The latter
tells {\tt DoPHOT} this is a ``warmstart'', the former informs the program that
this is the special sort of warmstart which assumes fixed positions from
a template file.   You will note two very special points about this final
run of {\tt DoPHOT}.  First, the initial threshold is 35.44!!  Since you are now
using positions from a (deep) template file, there is no need to search
for new stars at a high threshold. In fact, this final run has been 
set up so that {\tt DoPHOT} will only search once for any possible stars not
in the template file, but present in the program frame ({\it e. g.} 
variables, objects beyond the edge of the template frame, {\it etc.}).
Second, the final list contains 2458 objects, compared to the 1838 objects
in the list generated by the `straight' run of {\tt DoPHOT} without a template
file.    To be sure, not all the objects in the output list are 
reliable stars, but a number of objects that would not have been
split as two or more objects on the program frame, were treated as such 
because of the independent positional information from the template list.
Note too that some stars have {\it negative} fluxes.  This is OK, and 
simply represents cases where a star in the template frame was not 
recovered in the program frame (cosmic ray? very faint star? variable?).
The errors for such stars are negative, reflecting the fact that the error
is computed as 1.086*(($\Delta$ Flux)/Flux).  Upper limits may be computed
from the reported flux and the reported magnitude of the error.
For certain applications, this might be quite useful.
One final point:  you should have noticed that this final run of {\tt DoPHOT}
was somewhat faster than the original reduction of the program frame.

The best way to understand what files are needed is to study the parameter
files mentioned above.  The nomenclature of the file names has been 
designed to help you understand how the various files interact.  Although
the procedure described above is a bit cumbersome for a program designed
originally to run in an automatic manner, it provides a method of 
using a pre-determined position list in {\tt DoPHOT}.  We hope to try to 
implement a more automated version of the fixed-position option, but making
it bullet-proof will probably not be simple.

\centerline {\bf 11.3   Median-Filtered Background Model }
  
The object finding algorithm relies on a MODEL sky for initial 
estimation of the background when finding new objects. A candidate
object must then have sufficient signal to noise relative to a local 
average sky computed from the actual image data to be included in the 
object list for subsequent processing. If the model overestimates the 
background in the vicinity of an object, the initial object recognition
may fail, and the object will not be found. For images where the
background is not well described by a PLANE or HUBBLE model, 
it is possible to describe the background by median filtering the data 
image with a filtering box. The size of the filtering box corresponds
to the minimum length scale of features that can be tracked by this 
model, while features smaller than the box are suppressed. By keeping 
this box a little larger than the size of a star (say 2*FWHM), an 
adequate model of the background can be realized that tracks irregular 
backgrounds on scales larger than those for stars.

Note that just as for the PLANE and HUBBLE models, this option 
uses the background image as a MODEL for initial background estimation 
only:   ALL PHOTOMETRY IS DONE ON THE DATA IMAGE.

This procedure is invoked by setting the SKYTYPE parameter to 
MEDIAN. A background model picture is then created before the first pass
at the top threshold level. It may take some time to compute this image.
It should not be necessary to compute this model at every threshold
level: it is computed once again at the last but one pass after all the
bright objects have been subtracted. If you are really, really unhappy 
about this and want to change the frequency of updating the model, 
you will need to change the relevant controlling code, all of which is 
in the main DOPHOT module.

The median filtering routine itself has been tailored to the needs
of DoPHOT. We have used an algorithm that is fairly fast. It will use 
only legal values of the data between ITOP and IBOTTOM (parameters), and
will make appropriate compensations in saturated regions and `holes'.
Images with smoothly varying backgrounds are 
filtered much faster than those where the backgrounds fluctuate wildly.
There are several associated parameters you will need to set when
using a MEDIAN skytype.  These are {\tt JHXWID}, {\tt JHYWID}, {\tt MPREC},
and {\tt NTHPIX}.  Their function are described in Appendix A.

\centerline {\bf 11.4  EMPIRICAL Point Spread Function}

The principal change in version 3 is the addition of empirical PSF
fitting to the analytic PSF fitting of versions 1 and 2.  The
empirical fitting proceeds in parallel with the analytic fitting,
starting at a specified threshold (which ought to be somewhat lower
than the initial threshold, say a factor of 10, to give the program
some time to clean up the template star).  The empirical PSF, rather
than the analytic PSF, is used for the creating the star subtracted
image (with the significant exception that the analytic PSF is
subtracted from the template star itself).  The empirical template may
be written out as a subraster for examination.  The analytic PSF is
used for those objects classified as galaxies.

Version 3 includes a new function which interpolates values from
the empirical PSF using a 12 point scheme which is third order
in displacement.  It represents a compromise between using only
the nearest pixels (which clips peaks and fills valleys) and using
distant pixels which may be dominated by noise.  
We are anxious to receive feedback and comments from users regarding 
this particular approach to interpolation, which is different from 
the commonly used one of first subtracting an analytic function, such 
as a Gaussian, and then interpolating quadratically.

In principal empirical PSF fitting ought to produce better photometry
and cleaner star subtracted images.  Version 3 therefore permits a
different (and presumably smaller) residual noise factor to be used.

We have been slightly inconsistent in not using the empirical PSF in
splitting suspected double stars, though it might be especially useful
for this purpose.

For the moment a single star (cleaned of its neighbors) is used as the
template.  It is chosen automatically, starting from the brightest
star and working down.  One can ask that the n brightest stars be
skipped.  One can choose the size of the subraster required.  The
larger the subraster the more stars will be disqualified, either
because they lie close to the edge of the image or because there are
bad pixels close to the center.  Single bad pixels far from the center
are smoothed over by linear interpolation.  Adjacent (same row or same
column) bad pixels cause a star to be rejected.  
Alternatively, one can {\bf specify} the empirical PSF star manually
by giving its x, y and approximate central intensity.
The code was written
with an eye to future construction of the template from more than one
star. The use is controlled by new parameters THRESHEMP, N\_EMP\_SKIP, 
EMP\_STAR\_X(YZ), EMP\_PSF\_BOX, EMP\_REJ\_RAD\_SIG, and EMP\_SUBRAS\_OUT. A 
fuller description of what these parameters do can be found in Appendix A.

The empirical central intensities and positions are scaled and shifted 
by the best analytic fit to the template star, so as to make the two agree.
Thus reported positions and magnitudes from empirical fitting have 
{\bf no systematic difference}. The version 3 output files in COMPLETE and 
INTERNAL modes have 5 new additional columns with the results from 
empirical PSF fitting. The results from analytic fitting are also reported 
(in their original columns). Note that even the analytic fitting 
benefits from the improved object subtraction when empirical fitting 
is invoked. The output columns are described in detail in Appendix B.

An example of using the empirical PSF mode is provided: see the modified 
parameter file `pm' provided with the test image (in 
the verif\_data subdirectory). 















