\centerline{\bf 8. GUIDELINES FOR PROGRAMMERS}

\centerline{\bf 8.1 DoPHOT 4.0 Improvements from Fortran}

DoPHOT is now in C, but it was converted modularly from the
Fortran 77 subroutines.  Thus, perhaps to a C expert's chagrin,
it reads and executes just like the Fortran versions.  And just like 
in the Fortran version, comments explaining the logic within the
code are sparse. However, there are improvements over the
Fortran version which have made the transition worthwhile, at 
least for the person who converted it.

\centerline{\bf 8.1.1 Obsolete cdiskio Routine Replaced with CFITSIO}

No more cdiskio that uses obsolete (though admittedly still 
functional) read write functions.  The tradeoff is dependence 
on the CFITSIO library.  I am sorry to those who hate when 
software comes with ``and add this library/package� strings, but I 
was disinclined to reinvent the wheel when such a nice round one 
was available.  If you are desperate for a version without CFITSIO, 
email rsobel@mit.edu for an old, but clean copy of Dophot with 
limited features but no CFITSIO dependence.  However, CFITSIO 
is very nice, and I encourage you to just get it working and use the
current and maintained DoPHOT version with all the features.

\centerline{\bf 8.1.2 \#defines confined to {\bf structs/tuneable.h}}

All \#defines (in Fortran, �parameter�) are confined to the header 
file {\bf structs/tuneable.h}.  No others hide throughout the program.  
To developers, please keep it that way.

\centerline{\bf 8.1.3 Commonblocks Replaced by Structs}

Common block variables are now anonymous structs.  While these 
are seemly equal in their evil, there are subtle improvements which 
I will lay out in this section.

All struct definitions exist in a subdirectory called {\bf structs}.  If a 
function uses struct ``thisone'', a dedicated file with the struct called 
``thisone\_struct.h'' exists in the {\bf structs} subdirectory, and all routines 
that use that struct must �\#include �thisone\_struct.h'' at the top of the 
routine.  This structure makes (1) finding all routines that use a given struct 
very easy because the word �thisone\_struct� will appear in every routine
where the struct is used, and nowhere else.

Additionall every time a variable is used in a routine, it is explicit which 
struct it belongs to, the variable type, and whether or not the variable 
is changed in the routine.  This is done differently for lower level and higher
level subroutines.

For lesser routines, the struct variables are renamed to a convenient 
local variable name at the beginning of the routine and then 
reassigned at the end if changed.  This makes both the type explicit 
and if the struct variable is changed in the subroutine.  And when time 
(and gold and pixie dust�) allows I will go through and make a comment 
in the beginning of each routine and corresponding header file stating in 
words which variables are changed.  

For higher level routines ({\bf shape, isearch, improve, dophot}) 
where the struct variables are changed frequently in the routine and 
their updated values are needed by the subroutines called therein, 
the variable types and the fact that they�re used in the routine are 
mentioned in an appropriately commented area at the top of the 
routine before the routine `substance� begins.  Assume the struct 
variables are changed throughout all higher level routines unless 
otherwise stated.

Tuneable variables, tune\#\#\_.variable, are ALMOST never changed 
except in {\bf tuneup.c} where they are defined from the parameter 
files.  The exception is in the routine {\bf bestab.c} where there are angry 
comments letting developers know that this is happening and is 
supposed to happen, although it�s uncharacteristic of the tunable 
variables and will probably be changed at some point for consistency's 
sake.  Changes and lack of changes to the tuneable variables aren't a 
new feature of the C version, but worth mentioning since it is angrily 
commented in the code.

All struct {\it arrays} are actually pointers.  2d arrays are pointers to 
pointers.   Their length is not fixed in the struct definition itself, but rather 
malloced in each run in {\bf tuneup.c}, or at the very beginning of 
{\bf dophot.c}, even if the theoretical length of the array is fixed. These 
ARE THE ONLY TWO PLACES WHERE MEMORY FOR STRUCT 
ARRAYS IS ALLOCATED.  Developers, please keep it that way.  The
strict locations for memory allocation should make it easier when 
swapping in PSF models, or debugging segfaults generally.  Also, here 
is a good place to mention for those who care, the code is Valgrind 
clean.  

In a future version I should convert the anonymous structs into proper 
C structs.  Their current shy nature is a hold over from the conversion 
from Fortran 77, which happened subroutine by subroutine, and thus 
required that the C structs were able to communicate with the Fortran 
common blocks.

\centerline{\bf 8.1.4 No EQUIVALENCES}

The Fortran version of  DoPHOT had EQUIVALENCE statements which 
took an array of two short ints (integer*2) and treated them as one float 
(real*4) to be passed into structs (common blocks) that accepted floats.  
This was useful for passing around two numbers within existing structs, 
but is not really supportable (read: unclean).  Remnants of the 
equivalences in C (bit math) are now gone.  Short ints never become floats 
unless recast as such.

\centerline{\bf 8.2 DoPHOT 4.0: Things Unimproved from Fortran}

For the sake of fairness, I should lay out those aspects of the code that
are as yet unimproved from Fortran 77.  For the first, I made a time call.  
For the second, I learned better, but too late.  

\centerline{\bf 8.2.1 Commenting}

The Fortran code logic had very little commenting.  The C version has 
the same logic and thus the same lack of commenting.  The few bits 
which are �new� in C, rather than just translated, {\it are} commented.

\centerline{\bf 8.2.2 Naming Conventions}

Fortran naming conventions and capitalizations are carried over from 
the old code, especially in the smaller subroutines.  C conventions are 
not used.  This flaw will be slowly rectified as new versions are 
released.

\centerline{\bf 8.2.3 Pointer Passing}
Fortran passes everything by pointer, so so does the C version of 
DoPHOT, even if the variable in question is not changed in the routine.  
While this flaw has been fixed in a select few subroutines, it should be 
fixed more globally in a subsequent version to make it more explicit 
which variables are and aren�t changed in routines.


\centerline{\bf 8.3 Keeping the Point Spread Function Hidden}

When DoPHOT was written, a deliberate effort was made to
keep the details of the point spread function hidden from
most of the higher level subroutines.  It was hoped that
this would make it possible to use DoPHOT with very
different point spread functions.  The authors were only
partly successful in approaching this ideal.  Some
subroutines work even when the number and order of the
parameters are changed.  Others work if one changes the
dimensionality of the parameters (switching, say from
central intensity to log central intensity), but not if the
order of the parameters is changed.  Still others require
intimate knowledge of the details of the PSF (e.g.  the
subroutine which makes a first guess of the double-star
separation when fitting two ``typical'' stars to a single
``big'' image).

Below is what you need to know and do when you make a new model, 
referred to here as NEWMODEL.  This guide is somewhat step by step, 
but will still require some user knowhow and should not be attempted 
by the novice.  Moreover, this guide is good only for models which add
additional parameters to the existing core 7 (SKY, X position, Y position,
Peak Intensity, sigma\_x, sigma\_xy, sigma\_y) of the pseudogaussian
model PGAUSS supplied with the code and/or change the space in which 
they are fit (to log space etc.).  If one wishes to fit a model with {\it fewer} 
parameters or change the order or nature of these core 7 parameters, this 
guide will be helpful but incomplete.  However, we feel that a sky
value, position, intensity, and some elliptical shape parameters are 
necessary to most models and thus this guide will be helpful for many users.

The instructions will start from the most seeming peripheral and end with the 
most core.  This is by design to maintain usability of the code.  If you do get 
a new model working, consider it to be worthwhile,and are willing to share, 
please submit your proposed updated to github!

\item{1.)} Change {\bf params\_default\_c} to reflect the newly available 
	model type. Keep the default as 'PGAUSS'. Note: DoPHOT only 
	checks for model type by reading the first five characters of a model name, 
	so make these characters unique to your model, (i.e. EXT\_PGAUSS, not 
	PGAUSS\_EXTENDED). The relevant line in the {\bf params\_default\_c} 
	file should now read something like:

	PSFTYPE = `PGAUSS'        PSF type: (PGAUSS, GAUSS, NEWMODEL)


\item{2.)} Update {\bf tuneup.c} to recognize NEWMODEL as a new model
	type.  Additionally, make certain that it updates tune4\_.npar and 
	tune4\_.nfit2 to the appropriate number of shape parameters of your model, 
	as well as tune15\_.ava[] to the appropriate average shape values. Last, 
	ensure that the appropriate number of tune15\_.acc values are drawn from
	the parameter files.
	Remnants of a SERSIC model with 8 shape parameters exist in the code, 
	though the model itself was deemed far less worthwhile than the PGAUSS 
	model and therefore abandoned.  The SERSIC model switches can be used 
	as a guide within the code for necessary updates for new models.

\item{3.)} Add the header file for the NEWMODEL to {\bf dophot.c}.  The header
	file must match the PGAUSS header file {\bf structs/pgauss.h} exactly 
	except for the defines and the name of the relevant `2d' and 4d' functions.
	The `2d' function is for isolated objects.  The `4d' function is for double stars.
	These functions needn't have the same form.

\item{4.)} In {\bf dophot.c} add the model to the list of options which can be switched 
	by the model flags.  	This statement will tell DoPHOT to use your model 
	everywhere an analytic model is called for.  If newmodel2d\_ and newmodel4d\_ 
	are the names of your analytic models for single objects and double stars that you 
	had declared in the header file, your statement should look like:

	if (strncmp(flags[0], "NEWMODEL", 5) == 0)\{

	\indent \indent	model2d = \&newmodel2d\_;

	\indent \indent	model4d = \&newmodel4d\_;

	\}


\item{5.)} Modify the files which generate INCOMPLETE, COMPLETE, and SHADOW 
	outputs to display the fits of your new variables (if there are new variables).  
	These files are {\bf sumout.c, stdotpt.c} and {\bf shdout.c} respectively.  The type of
	output should be toggled by a switch on the flag, and we recommend that the new
	variables are displayed immediately after the sigma or semi-major and tilt fit 
	variables where appropriate.  If you follow the next steps, your new variables will 
	likely be stored in STPARR[7-10] (or more) in this routine, so feel free to output 
	these variables if you're intending to follow the remainder of this instruction guide.  
	(Actually struct 2darray starlist\_.starpar contains the fit parameters for all stars.  
	The 1d array corresponding to the parameters for each output star are passed to 
	the output routines as STPARR.) Again, look for the SERSIC example for a guide 
	within the code.

\item{6.)} Modify the files which read INCOMPLETE and COMPLETE warmstart files, 
	namely {\bf suminpt.c} and {\bf stdinpt.c}.  They should read files of the type you
	just designated as output and put the new variables in STARPAR[7-10] (or more) 
	as appropriate.

\item{7.)} Write the model function {\bf newmodel.c}.  DoPHOT uses models in two 
	ways, for fitting shape parameters to found 
	objects and for subtracting off the models of objects once appropriate shape 
	parameters are found.  Therefore each model must take four parameters: 1.)
	a pointer to an x,y position of a pixel in the field at which the model value is to be 
	computed, 2.) a pointer to the array of fit parameters, 3.) a pointer to the array of 
	derivatives of the fit parameters which are only needed
	for the fitting routine, and 4.) a pointer to an int, fitcall\_ptr, which can be used to
	tell the function if it is being called by the adding or subtracting routine 
	{\bf addstar.c} or the fitting routine {\bf chisq.c}.  The function must return the value
	of the analytic function at the position of the pixel passed in the first parameter.
	There is a fifth passed parameter, m\_ptr, which remains for historical reasons but 
	which has no utility in any of the supplied model routines.	
	With these parameters passed to your fitting function, you should be able to write
	the analytic model of your choice.  Follow {\bf pgauss.c} as a guide for the set up
	of both the one star and double object functions and necessary parameter updates.
	The fit parameters 'a' passed to the routine are taken from the struct starlist\_.starpar
	or fitarrays\_.a depending on if the analytic model is being subtracted/added to the 
	image or fit by {\bf chisq.c}.  If you have followed the guide so far, your additional
	parameters should appear there after the core 7 parameters of the 
	pseudogaussian model (beginning at a[7]).  The same goes for the derivative 
	matrix 'fa' which is taken from the struct fitarrays\_.fa.
	
\item{8.)}  Update the {\bf onefit.c} and {\bf twofit.c} chisq wrappers for the fit of the analytic 
	model. The onefit.c wrapper was created to support the fitting of models in a different 
	parameter space than that in which they are subtracted from the images and 
	displayed to the user.  Twofit.c was created to minimize the number of free 
	parameters used in the fitting of double objects, but it also allows the same parameter 
	space switching for fits.  The one star fitting functions and double star fitting functions 
	needn't be the same. 
	Regardless your model type, you should check these wrappers and make sure they 
	are still valid your model.  If you have chosen to fit your 
	model in a different space than that in which you add and subtract it from the image, 
	you will have to change your variables to the fitting space and update your covariance 
	matrix back from that fitting space after the fit in these two functions.

\item{9.)}  Add your new model, {\bf newmodel.c} to the {\bf Makefile}.  Compile the code.

\item{10.)}  Test that the PGAUSS model still works on the {\bf verif\_data} images.  
	You should still get exact results.

\item{11.)} Test your model.  

Troubleshooting.  If you run into a segfault, check NPMAX in {\bf structs/tuneable.h} to 
ensure that it is greater than or equal to the number of parameters in your new model.
Check that MMAX is 3+ NPMAX.  If either of these conditions does not hold, fix them,
remove all .o files and remake dophot.  
	


\centerline{\bf 8.4 ``Typical'' Parameters and the Shadow List}

The output object list uses the ``typical'' shape parameters
for the listed stars.  But those typical values are computed
from actual values determined from the individual stars.  A
``shadow'' list with these individual values is stored in
memory, and is used to compute the ``typical'' values.

The errors in the shadow parameters are also stored in
memory.  The simplest ``typical'' value is a weighted mean
of the parameters.  Some versions of DoPHOT (but not this one) 
allow the shape
parameters to vary with position on the chip, in which case
the shadow list and the associated errors are used to
constrain models for that variation.


\centerline{\bf 8.5 Magic Values and the Noise Array}

DoPHOT works on an image which is stored in a (as of 4.0)
32 bit signed integer array. The noise is stored as a variance 
(the square of the noise) in a 32 bit integer array.  In 3.0 and 
earlier the data was stored as a 16 bit signed integer array.  In 
a still earlier version of DoPHOT, the noise itself was stored 
as a 16 bit integer, but the repeated additions, subtractions 
and square roots had round off errors which led to negative 
variances.

In the routine which computes the noise array, image pixel
values are compared to user-specified upper and lower
limits.  When the data value lies outside this range, the
variance for that pixel is set to a ``magic'' value, $2^{31}
- 1$. The routine which extracts a subraster from the image
for fitting with an object model does not include pixels for
which the noise array has this magic value.  If the
subtraction (or addition) of the model for an object ever
causes the data value in the object-subtracted image to be
smaller than a \#defined OBLITVAL set to $-60000$ or greater 
than the user-specified max obliteration value CENTINTMAX,
the variance in the noise array is likewise set to this
magic value, and the pixel is henceforth ignored.  When a
subraster around a saturated star is ``obliterated,'' the
associated elements of the noise array are likewise set
equal to this magic value.

\centerline{\bf 8.6 First Guesses}

DoPHOT uses an iterative nonlinear weighted least squares
fitting for estimating the parameter values associated with
the various models it uses.  The particular nonlinear least
squares scheme converges rapidly if the current guess of the
parameter values is close to the best value.  But one needs
a good first guess -- a bad first guess will leave the
routine through wandering the valleys of the goodness of fit
parameter, searching for a minimum.  For this reason the
shape parameters for individual objects are saved in the
above mentioned ``shadow'' lists, for use as starting values
on subsequent fits.

The initial guesses which have proved the most difficult to
obtain and the least satisfying, both esthetically and as
measured by ultimate convergence, have been the choices of
the separation and central intensity ratio for the two
components of a double-star.  One knows the footprint of the
typical star, the ``oversized'' footprint of the object to be split, and
the positions and brightnesses obtained, respectively, using
the typical footprint and the best fitting footprint.  The
initial brightness ratio is arbitrarily taken to be 2, with
the brighter object taken to be closer to the position
obtained using the typical profile.

The separation vector is taken to be along the position
angle of the oversized footprint.  While this is reasonable
for a typical spherical footprint which is circular, one
could imagine a situation in which the ``typical'' footprint
is elliptical and the ``oversized'' footprint is more nearly
circular, but with its elongation perpendicular to the
separation vector.  Modifications which would help in this
situation have been suggested but have not been incorporated
into the present version.

\centerline{\bf 8.7 Nonlinear Weighted Least Squares}

At the core of DoPHOT lies an iterative nonlinear weighted
least squares routine in which Bevington's (1969)
implementation of Marquardt's algorithm has been imitated.
There are user-specifiable parameters which control the
convergence criteria, expressed either in terms of relative
accuracy, suitable for intensities and some shape
parameters, or absolute accuracy, which is more appropriate
for positions.  The user can also specify extreme upper and
lower bounds beyond which the routine declares itself to be
hopelessly lost in parameter space and gives up.  This last
feature permits the DoPHOT to recover relatively gracefully
from bad initial guesses and badly confused object
configurations.

\centerline{\bf 8.8 Data for PSF Fitting}

Since pixels can be excised from the image for a variety of
reasons, not every pixel contained within the requested
subraster is passed to the fitting program.  The $x$ and $y$
coordinates for each usable pixel, computed with respect to
the center of the subraster, are passed to the fitting
program, along with the data value for that pixel.  The
fitting program was written to fit a scalar function of
scalar variable, both of which are represented as 32 bit
floating point numbers.  Since the subraster is centered on
a pixel, the pixel coordinates can be represented as two
``short'' 16 bit integers.  These numbers are recast to
floats for fitting, and NO LONGER EQUIVALENCED in 4.0

\centerline{\bf 8.9 Switches versus Custom Versions}

A limited number of switches are provided which allow the
user to choose among image formats, 
among parameter output formats, and among models
for the background sky.  Switches both complicate and slow
down the program, but they avoid the need for multiple
versions of subroutines.  As a rule of thumb, switches
should be incorporated only when one expects them to
get frequent use.  The authors suggest that the user modify
the program rather than incorporate new switches.

\centerline{\bf 8.10  Variable Point Spread Functions}

There are versions of DoPHOT which allow the PSF to vary as
a function of one or both of the positions within an image.
The subroutines in DoPHOT are small and modular.
At most four of these must be changed in order to
accommodate a variable PSF.  Since the model for the
variation of the PSF across an image will depend upon the
details of the instrument used, the present version of
DoPHOT incorporates only the simplest of models, a constant
PSF.

\centerline{\bf 8.11  Subtracting a Smooth Image}

There are cases in which the user will want to subtract a
smooth model for the background light, but to keep the
photon statistics as they were prior to subtracting the
smoothed model.  There are several ways in which one might
do this, but one method that has been somewhat  successful involves subtracting
the model of the smooth background as part of the
preprocessing, but then using that smooth model to insure
that the photon statistics of the noise array are correct.
One could imagine running DoPHOT on an image, finding the
brighter objects, smoothing the object subtracted images,
subtracting them and iterating.

\centerline{\bf 8.12  Analytic/Empirical PSF }

It is the authors' suspicion that the greatest losses
involved in the use an analytic, as opposed to an empirical,
point spread function are associated with the
object-subtracted image.  If the residuals were smaller, one
could obtain more complete identifications, and better
photometry and classification.  

In the Version 3 release, DoPHOT was extended to include 
the option of using an empirical PSF. 
Tests show that the residuals after subtraction
are dramatically smaller, and the number of `outliers' in 
photometry of severely crowded fields is significantly 
smaller. A fuller account of this implementation is 
given in section 11 (Enhancements).

%One might achieve this
%without sacrificing the speed and simplicity of analytic
%point spread functions if one used a hybrid scheme,
%computing an empirical PSF for the purpose of creating
%better object-subtracted images but continuing to use an
%analytic PSF for the purpose of obtaining positions and
%magnitudes.  Such a hybrid scheme is {\it not} implemented in
%the current version of DoPHOT.




