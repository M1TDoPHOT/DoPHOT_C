\centerline{\bf 1. PREFACE}

DoPHOT is a computer program designed to search for
``objects'' on a digital image of the sky and to produce
positions, magnitudes and crude classifications for those
objects.  Digital images, like the research programs for
which they are obtained, vary enormously, and no single
computer program is likely to handle all cases equally well.
The particular application for which DoPHOT was designed,
and to an extent, optimized, involved large numbers of
poorly sampled, low signal-to-noise images for which a fast,
hands-off approach seemed desirable.  The scientific program
for which it was written required photometry of, what at the time,
seemed like a large
number ($10^5$) of relatively bright stars, with an accuracy
of 0.03 mag or better.

Anyone considering using DoPHOT would do well to think
carefully about using some other program (see, for example, the
various programs described in the proceedings of the 1st ESO/ST-ECF
Data Analysis Workshop edited by Gr\o sbol, Murtaugh, and Warmels)
or writing new code designed
specifically for the application at hand.  The user who
chooses DoPHOT may sacrifice completeness in the
identification of objects, and accuracy in their positions,
magnitudes and classifications.  This is especially likely
if DoPHOT is used on data sets (or for research programs)
which are very different from those for which it was
originally designed.  Some users may, despite these losses,
find it advantageous to use DoPHOT.

DoPHOT was written and modified by individuals more
interested in expeditiously carrying out their own research
than in producing a black box requiring minimal
understanding of its contents.  While some attempt has been
made to make the present version less user-hostile than
earlier versions, the program is not designed to check
whether the input data are ``reasonable", or whether the
output is ``reasonable".  The necessary ``reason" must be
supplied by the user, giving careful thought to the input
data and careful attention to the output results.  The
greater the user's understanding of the program, the more
likely he or she will find it useful.  Through this manual
the authors try to convey a thorough understanding of the
program, suggesting ways of monitoring results and pointing
out pitfalls to be avoided.

A second consequence of DoPHOT's origin as a program
intended for use only by its authors warrants mention.  In
the course of working with different data sets a variety of
unanticipated and frustrating difficulties were encountered
for which ad hoc solutions were implemented.  Many of these
``quick fixes'' used the programmer's equivalent of baling
wire and duct tape.  They are not pretty.  Moreover, while
many of the fixes adopted were adequate for the immediate
problem they were intended to address, they are far from
optimum.  New data sets are likely to reveal the
shortcomings of many of these fixes.

DoPHOT is a free-standing program, not part of any
``package,'' ``facility'' or ``system.''  The user is
therefore responsible for the needed preprocessing (bias
subtraction, ``flat fielding,'' and perhaps the editing of
unwanted parts of an image), and for the display of input
and output images.  Users who find that DoPHOT has been
incorporated into some local ``system'' should recognize
that it can be run outside that system, as a
non-interactive, batch job, which might avoid some of the
overheads endemic to such systems.  However, in its
current version in C, DoPHOT does now require the NASA
HEASARC CFITSIO library for reading and writing fits files.
We extend our apologies to those who detest library 
dependences. We just didn't want to reinvent the wheel on 
data io.

DoPHOT is composed of a great many relatively short modules,
allowing for their quick replacement with improved versions.
Since no two data sets present exactly the same challenges,
the modules have evolved, and we expect them to continue to
evolve, as new circumstances are encountered.  The user is
{\it encouraged} to improve upon and to customize these routines
and to advise the authors of improvements which they have
implemented and the results they have obtained.  The user is
{\it cautioned} to preserve a version of 
the code as originally distributed, to
guard against accidental corruption in the implementation
of genuine improvements.  It is the authors' experience that
while many of the difficulties encountered by users are the
result of genuine shortcomings of the program, many others
are the result of local modifications which have
unanticipated repercussions.  To help guard against such
accidental corruption, a test images and its associated
input/output files  are
distributed with the program, so that the user can check
whether the version he or she is using produces results
{\it identical}, not merely close, to those supplied.

DoPHOT's authors believe that astronomers are only beginning
to take advantage of the potential of digital images, and
that there is much to be had in the way of improvement.
Users of DoPHOT are therefore urged to think carefully about
which aspects of DoPHOT might be worth adopting in a next
generation reduction program and which aspects ought to be
abandoned for something better.

Having issued these discouraging warnings, some words of
reassurance are in order.  DoPHOT has been used with modest
success by a great many users, on traditional ``stare" mode
images, on time-delay-integration images, on Hartmann and 
Shack-Hartmann test images and even on the comparison arcs of 
echelle spectrograms.  The paper describing DoPHOT has been cited 
over 500 times.  Customized versions have been written which
allow (separately) for a point spread function which varies
with position within an image, for the classification of
objects as asteroid trails, and for taking proper account
of the noise associated with the subtraction a smooth
background (specifically an elliptical galaxy).  Photometry
has been obtained for many tens of millions of stellar images, with
relative photometry which on occasion reproduces to better
than 0.005 mag for a single measurement of a single star. At
least some of DoPHOT's users have found that the time
invested in learning to use it has yielded a satisfactory
return on their investment.

DoPHOT and its capabilities have been described in a paper: Schechter, Mateo,
\& Saha (1993), PASP 105, 1342. This manual acquaints the new user with 
running DoPHOT, and serves as a reference for all users alike. 
 
In the sections which immediately follow, DoPHOT is
described conceptually, with as little reference as possible
to variable names and subroutine names.  More detailed descriptions
of the parameter names and data formats are given in the two
Appendices following the main body of the manual.

The present manual describes DoPHOT 4.0, which is not much different from 
the manual for DoPHOT 3.0, except in the "Details for Connoisseurs" section.
This is by design.  DoPHOT 4.0 in C was built modularly from the DoPHOT 3.0 
routines in Fortran 77, and until intentional improvements were made, 
produced identical photometry. Although we have tried to change all instances 
of 1.0, 2.0, and 3.0 to 4.0, some cases may have slipped by the editors.
It is safe to assume that any references to DoPHOT 1.0, 2.0, or 3.0 really mean 
DoPHOT 4.0, and any file names containing {\bf 1\_0}, {\bf 2\_0}, or {\bf 3\_0}, 
should be read to contain {\bf 4\_0}.



