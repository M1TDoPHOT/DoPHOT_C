\centerline{\bf 2.  GENERAL DESCRIPTION}

A central feature of DoPHOT is the adoption of a {\it
model} for every kind of object which one seeks to identify
within the image.  The model for a cosmic ray is a single
high pixel.  The model for a star might typically be an
elliptical gaussian.  The model for a galaxy might {\it
also} be an elliptical gaussian, but one which is
signficantly bigger than those associated with stars.  The
model for a double-star is composed of two single stars.  To
classify an object, one chooses the model which fits the
object best.

With the exception of the model for a cosmic ray, these
models are usually specified in terms of analytic functions with
free parameters.  The output is
therefore a list of object classifications and parameters
for those objects.  Six parameters are associated with the
simplest models: $x$ and $y$ position within the image, total
(or central) intensity, and three shape parameters (length,
width and tilt).  While one can obtain better models with
more parameters, DoPHOT was written on the gamble (hunch)
that one might do reasonably accurate photometry and
classification using just these six parameters.

``Object'' is a somewhat slippery concept.  Operationally,
DoPHOT enters an object in its output list if it encounters
an array of pixels that are significantly brighter than the
surrounding sky when ``filtered'' through a model stellar
profile, or ``mask'' as it is called later in this manual.
The degree of significance and the size of that
array must be specified by the user (although DoPHOT
supplies defaults for these, and for all other user-specifiable inputs).

``Stars'' are a special kind of object (celestial point
sources) for which the three shape parameters are, to first
order, the same throughout the image.  In obtaining
magnitudes for stars, DoPHOT assumes that all stars have the
same ``typical'' shape.  DoPHOT determines that ``typical''
shape by fitting the brighter stars for the shape parameters
and taking a suitably weighted average.  In some versions of the code,
those shape parameters are allowed to vary smoothly across
the image.

If DoPHOT started out with no idea whatsoever of what the
``typical'' shape of a star was, then it would be unable to
decide whether the first object it encounters is a cosmic
ray, a galaxy or a star.  The user must therfore supply an
estimate of the FWHM in pixels of a stellar image.  DoPHOT
takes that estimate and multiplies it by 1.2 to insure that
none of the first stars encountered are accidentally
classified as galaxies, but the user must not be too sloppy
about the initial estimate of the FWHM.

The models for various types of objects are fit to
subrasters.  The sizes of these subrasters are
specified by the user or, alternatively,
determined by DoPHOT itself as some factor times the typical
FWHM of stellar objects.    In fitting the object
models the background sky level is treated as an additonal
free parameter.  Fitting to a uniform background corresponds
to taking a straight average of the pixels in the subraster
in the limit of an infinitely faint object.

This seemingly innocuous choice raises several interesting
questions.  If the subraster contains objects other than the
one being fitted, the parameters derived for the fitted
object will err systematically.  DoPHOT therefore {\it
removes} objects from the working image as they are
identified, subtracting the best current model for the
object.  This procedure is less likely to give poor results
if one first identifies the brighter objects, and then,
after removing them, identifies the fainter objects, much in
the spirit of the CLEAN algorithm (H\" ogbom 1974, {\it A. A.
Suppl.}, {\bf 15}, 417) used with
aperture synthesis data.

DoPHOT makes successive passes over the data, identifying
progressively fainter objects at threshold levels specified
by the user.  After each search the shape parameters for all
objects are redetermined, new ``typical'' values are
determined, and the stars are fit once again, now with the
new ``typical'' values.  Since the objects have been
subtracted from the image, they must be temporarily added
back, fitted, and then subtracted again.  All this is done
at considerable expense in computation, in the hope that the
newly derived parameters will be more accurate.

DoPHOT constructs and maintains a {\it noise image} which
provides weights for each pixel used in its non-linear least square fitting
subroutine.  It also produces, as part of its output, a
star-subtracted image.  The failings of the analytic model
for stellar objects are most obvious near bright stars,
where one often sees a consistent pattern of residuals from one
star to the next.  These positive residuals might be expected
to trigger the false identification of spurious objects.

To avoid such ``false hits,'' DoPHOT adds extra noise to the
noise array every time it subtracts a star from the image.
This is taken to be some fraction of the subtracted light,
by default 30\% (for the analytic model PSF)
 unless changed by the user.  Moreover, for
the purpose of calculating this extra noise, the linear size
of each image is taken to be bigger by some factor, by
default 1.3 (for the analytic model only), 
than its actual linear size.  The result of
this extra noise is that potential ``objects'' found in the
neighborhoods of previously identified objects are less
significant than they otherwise would be, and are therefore
less likely to be included in the output object list.
[The current version of DoPHOT implements an empirical PSF: since 
this is a much better match to the real stellar images, 
the extra noise added by default (can be changed by user) is only
3\%].

While this avoids ``false hits,'' it has the negative
consequence of making it more difficult to identify faint
stars near brighter stars.  The extra noise associated with
each object is temporarily subtracted from the noise array
when that object is temporarily added back to the
star-subtracted image for refitting.

Fitting models to subrasters offers the advantage that one
can solve for the parameters associated with a star even
when some pixels are missing, as might be the case if there
were a bad column, or if an object lies close to the edge of
an image.  DoPHOT also removes pixels from further
consideration whenever a cosmic ray is detected.  DoPHOT
will attempt to obtain magnitudes and shapes for objects as
long as some minimum number of pixels (which the user can
adjust) are still contained in the subraster.

Given the systematic pattern of residuals in the
star-subtracted images, one would expect that the total
fluxes derived from fitting the model point spread function
(PSF) to the data will likewise suffer systematic errors.
But to first order, one would expect to make the same
systematic error for all stars.  As an aid in correcting for
such systematic errors, DoPHOT calculates total fluxes in a
rectangular aperture of a size specified by the user.

Aperture magnitudes are very much more uncertain than fit
magnitudes because there is much more ``shot'' noise from
the sky inside the aperture than there is under the model
profile.  One must compromise between making the aperture
large enough  to collect most of the light from an
object and yet small enough to minimize the noise contribution
from the sky background.  While DoPHOT reports both fit and
aperture magnitudes, it is left to the user to compare these
two and to decide whether and how to correct the fit
magnitudes.  For some purposes (e.g. looking for variable
stars) the uncorrected fit magnitudes may be sufficient.

When the footprint of an object (the half intensity contour,
which in the present version is modelled as an ellipse) is
bigger than that of the ``typical'' star, one faces the
choice of classifying the object as a single star, a
multiple star or a galaxy.  DoPHOT does not do a very
thorough job of addressing this question.  If an object has
shape parameters which make it significantly bigger than a
star (by an amount which the user can adjust), it is
declared to be big and DoPHOT attempts to fit 2 ``typical''
stars to it.

The goodness of fit parameters for the fit of one big object
and two ``typical'' stars are then compared, and a decision
is made (based on a user adjustable parameter) as to whether
the object should be classified as a galaxy or ``split''
into two stars.  Stars which are ``split'' may be further
split, allowing at least in principle, for higher multiples.
If the default value of this user adjustable parameter is
used, the user will find spurious ``galaxies'' appearing
in globular cluster images and spurious ``split'' stars in
images with large numbers of galaxies.

An as yet unrectified shortcoming of DoPHOT is that it does
not maintain pointers linking such sibling objects.  The
photometry for these objects is obtained using the single
point spread function. It is more uncertain than it would be
if the siblings were fit as a pair because of the extra
noise added after the subtraction of the nearby sibling.

DoPHOT allows for a ``warmstart'' in which it reads a list
of objects and determines typical parameters from those
objects.  It allows the user to force object types for the
objects on the input list.  A special object type which is
sometimes useful is an ``obliteration,'' a rectangular
portion of the image which the program ignores.  While one
might prefer to deal with such regions by blocking them out
in pre-processing, including them in the object list gives
one a record of the areas not considered.  The program
automatically ``obliterates'' stars brighter than a
user-adjustable threshold.

























