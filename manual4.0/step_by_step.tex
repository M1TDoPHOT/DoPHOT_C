\centerline{\bf 6. INSTRUCTIONS FOR NOVICES}

This section assumes that you have successfully compiled
DoPHOT and would now like to try running it. Naturally
you would like to try running it on your own data.
{\it Avoid} that temptation, and instead, try running it
on one of the test images which are distributed with
DoPHOT. This will be very much easier and will teach
you some things about the program

\centerline{\bf 6.1  Running DoPHOT on the Test Images}

In addition to the executable file which resulted from
compiling and linking DoPHOT, you will need up to four additional
files (all of which are supplied with the test images):

\item{1)} an input image
\item{2)} a default parameter file
\item{3)} an image-specific parameters modification file
\item{4)} an input object list

The test images are written in ``FITS'' format, the parameter
files are written in a format which looks very much like FITS header
format (but isn't!).  
The object list is an ASCII table.  An input object list is only 
needed for ``warmstarts.''

The command needed to run the executable file will be different
for different computers, but likely `./dophot'.  When DoPHOT is 
finished, it produces up to six output files:
\item{1)} an object-subtracted output image
\item{2)} an output object list
\item{3)} an parameter report
\item{4)} a diagnostic log
\item{5)} a shadow file
\item{6)} an error file

Before you do anything else, run the program on the first of the test
images.  
You can do this by executing the dophot in the `verif\_data'
subdirectory, and supplying 'pm' as the image-specific parameters 
modification file.

Most computers have some means of comparing files, looking
for differences.  Under both the UNIX and VMS operating systems, that
command is ``diff''.  Compare the output object file with the output list in 
the `compare\_out' subdirectory supplied with the test images.  These 
should be {\it identical}, not merely similar.  If they are not, there is a very 
good chance that youare not working with the original version of DoPHOT.

Look at the output object list file.  Each row gives data
for a different object.  If you have chosen DoPHOT's external format for
the output photometry file ({\it i.e.} {\tt OBJTYPE\_OUT} = COMPLETE; see
Appendix A), then the columns are interpreted as follows:

\item{1)}  sequential object number, with the objects in the input
list at the beginning
\item{2)}  object classification, with 1 $\Leftrightarrow$ a single star, 
2 $\Leftrightarrow$ galaxy,
3 $\Leftrightarrow$ one sibling of a multiple,  7 $\Leftrightarrow$ a 
faint object for
which the shape was not determined and 8 $\Leftrightarrow$ an object which
could not be modeled and has, instead been excised.
Other object classifications are described in Section 7.
\item{3)}  $x$ position (x=0.5 $\Leftrightarrow$ centered on first pixel)
\item{4)}  $y$ position (y=0.5 $\Leftrightarrow$ centered on first pixel)
\item{5)}  ``fit'' instrumental apparent magnitude, computed from
the model of the object {$-2.5 \times \log_{10}$[total intensity in DN]).
\item{6)} formal uncertainty in the fit magnitude
\item{7)}  sky level (in Digital Numbers, DN)
\item{8)}  $a$, the long dimension (FWHM) of the footprint
\item{9)}  $b$, the short dimension (FWHM) of the footprint
\item{10)}  $\theta$, position angle of the footprint in degrees
\item{11)} ``extendedness'' parameter
\item{12)} ``aperture'' instrumental magnitude
\item{13)} uncertainty in the aperture magnitude
\item{14)} aperture sky value
\item{15)} difference between the aperture and fit magnitudes

The objects in the input list (if one was used) 
should be the first objects on
the output list.  The difference between the aperture and
fit magnitudes, while redundant, is a useful diagnostic.
This number should be the same (to within the quoted
uncertainties) for all stars.  The scatter in this quantity
should be small for the brightest stars, but grows rapidly
for the fainter stars.  While the fit magnitudes have
smaller uncertainties, and give good relative magnitudes for
objects within the image, an {\it average} correction to
aperture magnitudes is needed to compare measurements
obtained from different images and to calibrated frames using
external standards.

The values for the footprint ellipses are different for each
of the galaxies, but are identical for all of the stars.
There are versions of DoPHOT which allow the typical
parameters to vary in a smooth fashion across the image, in
which case the footprint parameters would not be the same.
For the case of ``obliterated'' objects the FWHM refer to the sizes
of the $x$ and $y$ sides of the corresponding rectangular 
obliteration boxes.

The parameter report file should be nearly identical to
the default parameter file, with the only differences
being the parameters specified by the parameter modification
file, or those changed when prompted by DoPHOT due to
invalid inputs in the parameter modification file.

The output, object-subtracted image file is a 32bit int FITS
format file.  Regions which have been
excluded from consideration because of bright stars or
blemishes will appear black if stars are displayed as bright
on a dark background.

If the galaxy model and point spread function were perfect, 
one would see only noise in the object-subtracted image.  
Since these are only approximate, even when using the 
empirical PSF, one sees a characteristic pattern in the
residuals at the positions of the objects.  If the user is interested 
in seeing the comparative improvement of the empirical PSF 
over the elliptical PSF, he should refit the images after commenting
out THRESHEMP, EMP\_STAR\_(X,Y,Z), and 
EMP\_SUBRAS\_OUT in the parameter modifications file.  The 
residuals on the star objects will now be larger as an analytic 
model is subtracted. 
 
An interesting exercise is to subtract the object-subtracted
image from the input image.  This generates a synthetic
image which contains only the information in the object
list, and is noise-free.  This is often a useful diagnostic,
since it can indicate, for example, when ``objects'' have
been incorrectly detected on the diffraction spikes of
bright stars.

The shadow file and error file contain the fitted shape parameter
information for each of the objects.  Unlike the output object file,
the shadow file contains the unique shape information from the 
fit of each star object rather than the average star shape subtracted 
from the image.  The error file reflects the same.  Details on the
row information can be found in Appendix A.

\centerline{\bf 6.2 Running DoPHOT}

If the user has two images of the same field, taken through
the same filter, he or she would be well advised (other
things being equal) to reduce these first, and to compare in
detail the completeness and the relative accuracy of the magnitudes
and positions obtained for the two images.

Before running DoPHOT, an image should have had any
electronic bias level subtracted and should have been
corrected for pixel-to-pixel photometric variations across
the image.  Any regions (bad columns, overscan, bloomed
images) for which the data are ``obviously'' a problem
should be masked off, either by setting the data values in
such regions equal to the lowest allowable value consistent
with the data format (often -32768) or by entering such
regions as ``objects'' in the ``input object'' file as
regions to be obliterated.  If one is too stingy about
masking off such regions, DoPHOT will consume a great many
CPU cycles finding spurious ``objects'' in the wings of
objects which extend beyond the masked region.

The ``modified parameters'' file will at the bare minimum need a
specification of the full width at half maximum (FWHM) of
the typical stellar image.  Better classifications will be
obtained on the first pass through the program if an
approximate sky level, good to 10\%, is also specified.

The user may specify file names for the input and output
image files, the input and output object lists, and input
and output parameter files and the output diagnostic log.
The very first thing that DoPHOT does is ask the user to
specify the name of the parameter modification file.  A valid file
{\it must} be specified because it is from the modified parameters file
that DoPHOT determines where the default parameters file exists.

Two user-specified numbers are used to compute the noise
array: the read noise (in photons) per pixel and the number
of photons associated with each digital number (DN) in the
image.  The number of photons per DN can be estimated by
looking at the pixel-to-pixel variations in high intensity
flat field images.  The read noise can be estimated by
looking at the pixel-to-pixel variations in low intensity
dark or bias images.

There are {\it two ways} in which the user controls the
faintest objects identified by DoPHOT.  Arrays of pixels in
the vicinity of potential objects are filtered through a
model stellar ``mask''.  Entries are added to the object list
only when the signal through this mask is significantly
greater than the surrounding sky.  This minimum
signal-to-noise ratio, measured in terms of Gaussian sigma,
is user specifiable.  Relatively few spurious objects are
detected when it is set to 5, but large numbers of spurious
objects pass the test when it is set to 3.  The calculation
of the noise passing through the filter depends upon the
noise in the individual pixels, which in turn depend upon
the read noise and the ratio of photons per DN.

Upstream of the filter, a decision must be made about which
groups of pixels are to be tested.  This is done looking for
{\it single high pixels}.  When a single pixel is
significantly higher than the sky, a group of pixels
surrounding this pixel is tested through the filter.  In
retrospect, the best single pixel test might have been a
simple signal-to-noise test.  Unfortunately, in the
present version of DoPHOT the test is more complicated.  The
pixel must have a value higher which is greater than the sky
by a pixel threshold and the noise level must be lower
than a specified fraction (by default unity) of this
threshold.

This pixel threshold is decremented in steps, starting with
a high value, so as to trigger on only the brightest stars,
and proceeding to the fainter stars.  The user should specify
both the highest and lowest value desired and the
logarithmic decrement between steps.  The highest threshold
should be chosen to trigger on the brightest few stars.  At
lower thresholds, great numbers of pixels pass the pixel
test only to fail on the more time consuming stellar filter.
The user should carefully choose a lower threshold value
that allows DoPHOT to trigger on mostly real stars 
and not waste too much CPU time on false alarms.

It is often the case that data values outside certain limits
are suspect and should not be used in estimating identifying
of measuring objects.  The user has the option of specifying
these limits.  Astronomical detectors often exhibit
complicated non-linear behavior when exposed to high light
levels.  There is a user specifiable upper limit to the
central intensity of the stars the program will attempt to
fit.  Brighter stars will be excised from the image, if the
user hasn't already done so.
Also, if the absolute data value of a given pixel is less than 
a constant (user changeable through parameter SNLIM -- set to 0.5 
by default) times the noise, 
the pixel will be ignored during PSF fitting. This can happen
if the image has been sky subtracted. We recommend that you use 
the original data so that the noise model is calculated correctly, 
but if you must subtract the sky, make sure SNLIM is set to zero
to avoid problems.

Having read the above paragraphs, the user should now look
at the default input parameter file distributed with the
source code and displayed in Section A.5 of Appendix A. The names
of the parameters and the comments next to them should
indicate which ones will need to be changed to accommodate
the user's image.  The user should create a modification parameters
file with values for these parameters (starting
with the FWHM) appropriate to the image.  Only parameters
that differ from those in the default file need to be
specified.  Before running DoPHOT, the user should make
sure, at the very least, that the FWHM of the stellar images
is specified and that the sky level, and the highest and
lowest pixel thresholds, as specified either in the default
parameter file or the parameter modification file, are
suitable.

Having run DoPHOT, the user would do well to mull over 
the output object list and to look at the original image,
the object-subtracted image, and the difference between the
two.















