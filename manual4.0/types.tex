\centerline {APPENDIX C: SUMMARY OF OBJECT TYPES}

\vskip1em

Type 1:  A `perfect' star. Was used in computing weighted mean shape
         parameters for stars.

Type 2:  Object is not as peaked as a star, and has been interpreted 
         to be a `galaxy'. Shape parameters were calculated specifically
         for this object.

Type 3:  An object found to be not as peaked as a single star was 
         interpreted to be two very close stars. This is one component 
         of such a close pair. Final fit to this component used mean shape 
         parameters for single star.

Type 4:  Failed to converge on a 4-parameter fit using mean single 
         star shape. See section 7.9.
         Photometry is extremely unreliable.

Type 5:  Not enough points with adequate S/N could be found around this
         object while attempting a 7-parameter fit to determine shape.
         Indicates a problem with the local environment, although a
         4-parameter fit with mean star shape succeeded. Photometry is 
         suspect, and this may not be a bona-fide star. See section 7.9.

Type 6:  Too few points with adequate S/N to do even a 4-parameter fit
         with mean star shapes. Object NOT subtracted from image.
         See section 7.9. 

Type 7:  Object was too faint to attempt a 7-parameter fit. No object 
         classification could be done, so it might be a faint galaxy 
         or two closely spaced stars. Photometry is OK (from successful 
         4-parameter fit with mean shapes) provided the 
         object is really a star. See section 7.4 for discussion 
         of star/galaxy discrimination for such objects.

Type 8:  These are obliterated regions due to data saturation or 
         identification of cosmic rays. See Section 7.8. The shape 
         descriptors are different for Type 8s than for other types:
         see Appendix B.
         
Type 9:  If an attempt to determine the shape of an object fails to 
	converge, it is classified as object type 9.  See section 7.9.
	Photometry is unreliable.
          

\vskip1em

Adding 10 to any of these image types means that the position was fixed
if {\tt FIXPOS} was set equal to {`YES'}.