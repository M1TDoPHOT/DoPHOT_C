\centerline{\bf APPENDIX A:  DoPHOT PARAMETERS}

DoPHOT contains a large number of user-specified parameters.
This should not immediately panic most users because in
practice, only a small number of these parameters typically
need to be adjusted.  Moreover, many
of the parameters can be allowed to scale with such quantities as the 
FWHM of the stellar objects on an image or the 
approximate mean sky level.  Many of the functions controlled
by DoPHOT's parameters have been discussed
in earlier sections of this manual, especially Sections 6-9.  
The purpose of this Appendix is
to describe the actual parameters used by DoPHOT
and to illustrate how a user can adjust them.  Some hints as to how
a user {\it should} adjust a given parameter are also provided where
appropriate.  {\it It is
very strongly recommended that users become acquainted 
with the material described 
in this Appendix before attempting to run DoPHOT on any
of their own images.}  If you adjust the parameters
blindly, you may get burned.  Used with care, however, they allow
DoPHOT to successfully analyze a wide variety of images.

\centerline{\bf A.1. A Brief Comment on Units}

``Digital Numbers'' (DN) refer to the counts in an image that is 
ready for processing by DoPHOT or any other photometry program.
DN can be converted to electrons via the detector's 
`gain' in electrons/data number.
DN are sometimes referred to as Analog-Digital-Units
(ADU), or Data Numbers (also DN).  DoPHOT does not care what the
scale of a picture is; consequently all $(x,y)$ coordinates are
expressed in pixels.  

\centerline{\bf A.2 The Parameter File}

When DoPHOT is started, it reads the default values of the parameters
from a ``default parameters'' file that is specified by the user or
in a ``modified parameters'' file.  The default parameters file suppled with 
version 4.0 of DoPHOT is called {\bf param\_default\_c}.  This file includes
{\it every} legal DoPHOT parameter and assigns a default value to
each.  The values given for the parameters listed in the default parameters 
file can be adjusted to suit the user's needs, but the names of the 
parameters cannot. {\it A default parameters file that includes all legal 
DoPHOT parameters must exist for} DoPHOT {\it to run.}  

The modified parameters file is designed to include only
the parameters whose values the user wishes to alter compared to what is
provided in the default parameters file.  Thus, let's say that you have
a frame where {\tt FWHM} = 3.2, not 2.0 as is listed in the default
parameters file (we haven't defined the parameter {\tt FWHM} yet, but
you can guess what it means).  The modification parameters file might
then only include a line with the new value of {\tt FWHM}; all other values
for DoPHOT parameters will then be taken from the default parameters file.
Of course, you can explicitly list any or all other parameters
in the modified parameters file, even if their assigned values do not
differ from what is in the default parameters file.  The basic rule is that
whatever values are assigned in the modified parameters file will be 
adopted, and the values of any parameters not assigned in that file will be
taken from the default parameters file.
Ideally, when you run DoPHOT the program will ask you only for the modified
parameters file name.  If all is well with this file and with the
data, there will be no more questions or input of any sort as the program
proceeds to reduce your data.  

The remainder of this appendix describes the rules that govern the
parameter files so that you can properly and reliably transmit your
the parameter values you wish DoPHOT to use.  The rules that govern the 
usage of default, modification, and output (see below) parameters files 
are identical so we will usually just refer to ``parameter files'' in a generic 
manner.

A typical line in a parameter file may look like this:

\noindent {\tt FWHM = 3.2\ \ \ \ \ \ \ \ Approx FWHM of objects in pixels.}

\noindent The important features of this line are:  (a) a {\it keyword}
-- in this case ``{\tt FWHM}'' -- that begins
in the first column of the line, (b) an
``equals sign'' character with at least one blank space immediately 
before and after
it, (c) a value -- in this case {\tt 3.2} -- that will be assigned to 
the variable associated with the keyword, and (d) a comment statement
that briefly describes the meaning of the keyword.  The comment is
optional.   Additionally, the equals sign is optional which will tell dophot
that the value assigned is NULL, however it is better practice to comment 
out NULL assigned keywords with an equals sign as the first character of 
the line.  DoPHOT parameter files look a little like FITS headers, but they 
are not identical, and the differences should be appreciated.

This example of a parameter file line illustrates the use of a
keyword to assign a numeric
value to a variable.  The variable {\it type} ({\it e.g.} REAL or INTEGER)
is still specified using the standard FORTRAN convention, even though
the code is now in C.  Thus, any keyword starting with a
letter between ``{\tt I}'' and ``{\tt N}'' (inclusive) is assumed
to represent an integer variable; otherwise the variable is 
assumed to be real.

That is, if it is not a character string.  The following line illustrates
an example of how one specifies a character string in a parameter
file:

\noindent {\tt IMAGE\_IN = 'A\_nice\_picture'\ \ \ \ \ \ \  \ \ \ Input image name.}

\noindent This line also includes a keyword that
begins in the first column, an equals sign with blanks before
and after, a character string (in quotes),
and a comment.  Again, the comment is optional, and lack of an equals sign
will indicate that the keyword is set to null, though it is better practice to
indicate that the line is a comment by putting an equals sign at the beginning
of the line. Unlike versions 3.0 and earlier, setting a character string equal to
a string of one or more spaces will NOT set it equal to NULL and may cause 
problems.  It is worth emphasizing that the character string {\it must} be 
surrounded by single quotes as shown although trailing blanks are fine.  
The valid FORTRAN convention of using two consecutive single quotes to 
represent an apostrophe is {\it not} supported in DoPHOT parameter files.  
There are no restrictions/conventions regarding the naming of keywords 
associated with character strings -- the quotes tell DoPHOT what it needs to 
know in such cases.

As shown in the examples above, comments can be appended to the end of 
a keyword line.  A full-line comment can be placed in any parameter file simply 
by placing the equals sign in the first column.  The following is a valid full-line 
comment:

\noindent{\tt =  This is a full-line comment.  The `=' is in the first column.}

\noindent Notice that the quotes and the righthand equals sign are valid since
the position of the lefthand equals sign identifies the entire line as a
comment.  Any line in a parameter file will be truncated at 200 characters when 
it is read.   These rules are summarized using full-line comments that are 
included in the file {\bf params\_default\_c} that is included with the code.  The
contents of the default parameters file is listed below at the end of
this appendix.

The following are examples illegal parameter file lines:

\noindent{\tt This is supposed to be a comment, but look where the `=' is!}

\noindent{\tt ThreshMin = 20.0 \ \ \ \ Keywords CANNOT mix upper and lower case,
unless of course you have defined new ones}

\noindent{\tt OBJECTS\_IN=`file' \ \ \ \ \ Illegal; gotta have spaces on BOTH sides 
of the `='.}

\noindent{\tt OBJTYPE\_IN = `INterNAL' \ \ \ \ \ Character strings CANNOT mix
upper and lower case, unless of course you have defined new ones}

\noindent  If the illegal lines shown above were  included in a parameter 
file, we'd hate to think what might happen.  You should be advised that
the authors have not tried to program DoPHOT's parameter 
input routines to cope with all possible contingencies, so be careful and follow 
the rules laid out in this section.

\centerline{\bf A.3  The {\bf structs/tuneable.h} File}

This file is used to pass variables between DoPHOT subroutines
using the C {\tt \#include} statement and it is not really designed to
be user-adjustable.  However, if (a) the number of objects measured on
a single image exceed 20000 DoPHOT's default array sizes 
will be too small to cope and the corresponding parameter {\tt NSMAX} 
should be changed here.  Of course, if memory is in short supply, the user 
can likewise decrease {\tt NSMAX} here.  Unless you really mean business, 
(i.e. are inserting a new model with a greater number of parameters than 
the default) {\it no other parameters in this file should ever be adjusted!}   

\noindent {\bf VERY IMPORTANT:} 
If {\bf structs/tuneable.h} is ever changed, 
{\it all} of the DoPHOT modules will have to be re-compiled or they 
will be unaware of the changes you have made in the include file.  
Makefiles to recompile all of DoPHOT are supplied with the code.  
Because Makefiles look for modifications, Unix users must first delete all 
the *.o (that is a lower-case letter o, not zero, not upper-case) files in their 
working DoPHOT directory first.

\centerline{\bf A.4 A Description of DoPHOT's Parameters.}

This section describes the parameters found in the parameter
files used by DoPHOT.  A full understanding of some of these
can be best attained by looking carefully at the code.   The
parameters described here are listed in the same order that
they are found in the parameter file.  This order is not
random.  Specifically, the most frequently-adjusted parameters
are listed first, with subsequent sections describing parameters that
are adjusted less frequently.

\centerline{A.4.1 Frequently updated parameters.}

Some or all of these parameters can be expected to vary from frame to frame.

\noindent {\tt FWHM} -- The initial guess of the 
full width at half maximum along the
major axis of a typical, isolated stellar object in pixels.  A modest
accuracy (say, 10-20\%) in the estimate of this is fine, but avoid 
providing FWHM estimates that are significantly too small.

\noindent {\tt AXIS\_RATIO} -- The initial guess of the
ratio of the minor axis divided by
the major axis for star objects.  This parameter is 1.0 for 
perfectly round objects.  This parameter rarely needs to be
changed from 1.0 unless the images are {\it very} elongated, say
for axis ratios less than about 0.7 or so.

\noindent {\tt TILT} -- The initial guess of the position angle
of the major axis with respect to the positive $x$-axis 
for a typical unblended stellar object on an image.  This
parameter has no effect if the axis ratio is 1.0.  The tilt is
given in degrees between $-$90 and +90, with the positive
y-axis at +90 degrees.

\noindent {\tt SKY} -- The initial guess of the mean sky value of
the frame in DN.  An accuracy of 10-20\%\ is fine, but 
a better estimate is advantageous.

\noindent {\tt NFITBOX\_X, NFITBOX\_Y} -- The sizes of the
sides of the fitting box
in the x and y directions, respectively.  These parameters can be 
automatically scaled
by the FWHM using {\tt AUTOSCALE} (see below).

\noindent {\tt MASKBOX\_X, MASKBOX\_Y} -- The sizes of the sides of
the mask box used for finding objects in the x and y directions,
respectively.  These parameters can be automatically scaled by
the FWHM using {\tt AUTOSCALE} (see below).

\noindent {\tt APBOX\_X, APBOX\_Y} -- The sizes of the sides of the
aperture photometry box in the x and y directions, respectively.  
In the current
version of DoPHOT, the {\it sky} measurement for the aperture
photometry is performed in a square annulus with outer side
lengths equal to twice {\tt APBOX\_X} and {\tt APBOX\_Y} in x and
y, respectively.  The inner boundary of this sky annulus is
defined by the aperture photometry box.  These parameters
are scalable by the FWHM using {\tt AUTOSCALE} (see below).

\noindent {\tt THRESHMIN} -- The value above sky 
of the lowest threshold that
DoPHOT uses to search for objects, in DN.   Thus,
if this parameter is 20, and the actual sky value is
450, the final threshold will occur at 470.  This parameter
can be scaled using the estimate of the sky value and the
readout noise of the detector (see {\tt AUTOTHRESH} below).

\noindent {\tt THRESHMAX} -- The maximum possible first threshold that
DoPHOT  uses to search for objects, in DN.  
Because of the function of 
the following parameter, {\tt THRESHDEC}, it is important to 
understand that THRESHMAX may {\it not} be the actual value used
for the first threshold pass.

\noindent {\tt THRESHDEC} -- The separation between successive
thresholds in units of 2.0 raised to the power {\tt THRESHDEC}.  Thus if this
parameter is 1.0, and {\tt THRESHMIN} = 20, and {\tt THRESHMAX} = 
1000, the thresholds (relative to the sky value) would be
640, 320, 160, 80, 40, and 20.  In this example, each level is separated by
a factor of $2^1$ from its neighboring levels.

\noindent {\tt THRESHEMP} -- If {\tt THRESHEMP} is less than {\tt THRESHMIN} 
(0.0 is default), it does nothing. For positive 
value V, once the threshold drops below V, empirical PSFs are fit. Ensure
that stars will have been found by the time the threshold drops below V.

\noindent {\tt MAX\_PERF} -- {\it new feature} The maximum number of type 1 stars that will be used to compute the weighted averages of shape parameters for star objects.  It is useful to set this number low only if the error estimates on shape parameters in new models are underestimated for dimmer objects or dimmer objects are likely to be misclassified as stars.  Otherwise the weighted averages will account for the error in the parameter estimations of the dimmer stars.

\noindent {\tt EPERDN} -- Electrons per DN for the 
detector.

\noindent {\tt RDNOISE} -- The readout noise of the detector in
electrons.

\noindent {\tt AUTOSCALE} -- This flag can only be set to 
YES or NO and if DoPHOT has any trouble reading the value of
{\tt AUTOSCALE} it will default its value to NO.  
If {\tt AUTOSCALE} is YES, then the box sizes described above are scaled by 
the FWHM using the parameters {\tt SCALEFITBOX}, {\tt FITBOXMIN},
{\tt SCALEAPBOX}, {\tt APBOXMIN}, {\tt SCALEMASKBOX},  and
{\tt AMASKBOXMIN}.  If {\tt AUTOSCALE} is NO, then the values
of the box sizes are taken directly from the parameter file 
without any scaling.
The scaling parameters used by {\tt AUTOSCALE} are described 
as follows:

\hang {\tt SCALEFITBOX, FITBOXMIN} -- 
{\it These parameters are used
only if {\tt AUTOSCALE} is} YES -- 
{\tt NFITBOX\_X} and 
{\tt NFITBOX\_Y} are set equal to the smallest odd integer
that is larger than 
{\tt SCALEFITBOX} times the object FWHM in the x and y
directions (these do not necessarily equal the value
of {\tt FWHM} described above), respectively.  If either or
both of these 
box sizes end up smaller than {\tt FITBOXMIN},
then the relevant fit-box size is set equal the nearest
odd integer larger than {\tt FITBOXMIN}.

\hang {\tt SCALEAPBOX, APBOXMIN} -- 
{\it These parameters are used
only if {\tt AUTOSCALE} is} YES -- 
{\tt APBOX\_X} and 
{\tt APBOX\_Y} are set equal to the smallest odd integer
that is larger than 
{\tt SCALEFITBOX} times the object FWHM in the x and y
directions, respectively.  If either or
both of these 
box sizes are computed to be smaller than {\tt APBOXMIN},
then the relevant fit-box size is set equal the nearest
odd integer larger than {\tt APBOXMIN}.

\hang {\tt SCALEMASKBOX, AMASKBOXMIN} -- 
{\it These parameters are used
only if {\tt AUTOSCALE} is} YES -- 
{\tt MASKBOX\_X} and 
{\tt MASKBOX\_Y} are set equal to the smallest odd integer
that is larger than 
{\tt SCALEMASKBOX} times the object FWHM in the x and y
directions, respectively.  If either or both of these 
box sizes turn out to be smaller than {\tt AMASKBOXMIN},
then the relevant fit-box size is set equal the nearest
odd integer larger than {\tt AMASKBOXMIN}.

\noindent As a rule of thumb, it is best not to use
{\tt AUTOSCALE} ({\it i.e.}, set it to NO) until you are quite 
familiar with DoPHOT.  Moreover, the default values for the
scaling parameters might be expected to vary among different
telescope/detector combinations, so be wary.

\noindent {\tt AUTOTHRESH} -- This flag can only have the 
values YES or NO.  Any illegal or unitelligible
value for this keyword in the parameter file
will cause this parameter to default to NO.
If {\tt AUTOTHRESH} is YES, then {\tt IBOTTOM} and {\tt THRESHMIN}
are scaled according to the sky+read noise
using the scaling parameters {\tt SIGMAIBOTTOM} and 
{\tt SIGMATHRESHMIN}.  If {\tt AUTOTHRESH} is NO, then
DoPHOT takes the values of {\tt IBOTTOM} and 
{\tt THRESHMIN} supplied by the user.  The threshold scaling
parameters controlled by {\tt AUTOTHRESH} are described
as follows:

\noindent {\tt FIXPOS} -- This flag can only have the values
YES or NO.  Any illegal or unitelligible
value for this keyword in the parameter file
will cause this parameter to default to NO.
If {\tt FIXPOS} is YES, then the program assumes you have
an input template file with the positions of objects as
described in detail in Section 11.1 and 11.2.  This template
file should be specified with the {\tt OBJECTS\_IN} parameter
(see below).  Be careful:  the fixed-position version of
DoPHOT is tricky to use and one should use this parameter
with caution.  If {\tt FIXPOS} is NO, then DoPHOT runs without
fixing any of the object positions regardless of what file
(if any) is specified by {\tt OBJECTS\_IN}.

\hang {\tt SIGMAIBOTTOM} -- 
{\it This parameters is used
only if {\tt AUTOTHRESH} is} YES -- 
{\tt IBOTTOM} is set equal to 
{\tt SKY} {\it minus} {\tt SIGMAIBOTTOM} times the sky+read
noise.  This  noise equals $[g s + r^2]^{1/2}/g$, where
$g$ = {\tt EPERDN} in electrons per DN, 
$s$ = {\tt SKY} in DN, and $r$ = {\tt RDNOISE} in
electrons.

\hang {\tt SIGMATHRESHMIN} -- 
{\it This parameters is used
only if {\tt AUTOTHRESH} is} YES -- 
The lowest threshold is set equal
to the sky+read noise times {\tt SIGMATHRESHMIN} where the noise
is defined as for {\tt SIGMAIBOTTOM}.

\noindent As in the case of {\tt AUTOSCALE}, be wary of using
the {\tt AUTOTHRESH} parameters until you become familiar
with how DoPHOT works without using this switch.

\noindent {\tt PARAMS\_DEFAULT} -- This keyword specifies 
the name of the default parameter file.  If DoPHOT cannot find
the file, it will be requested.  When you run DoPHOT, you must
create a `modification' parameter file that only needs to contain
the parameters you wish to change.  The values of any parameters that the
program cannot find in the modification parameter file will be
taken from the default parameter file.  Any bogus (or misspelled)
parameters in the modification file are ignored.  However, a default
parameter file with 
{\it all} legitimate parameters {\it must} exist or DoPHOT will not
run.  Such a file is provided with this release of the program, and
it should be guarded carefully.

\noindent {\tt PARAMS\_OUT} -- DoPHOT records the values of all
the parameters it used for a given run in the file listed with
this keyword.  If this line is blank (i. e., just two single
quotes or
any number of blanks surrounded by single quotes),
then DoPHOT assumes you
don't want such a file and none is  saved.

Because DoPHOT does not check the legitimacy of all the required
files read from the default and/or modified parameters file until
after the output parameters file is written, some
invalid file names can end up being saved in the output parameters file.
This bug was identified too late to correct in version 2.0 of 
DoPHOT.

\noindent {\tt IMAGE\_IN} -- This keyword identifies the input
image name.  It {\it must} exist, naturally, and if the name
in the parameter file does not correspond to any existing file, 
then DoPHOT will query the user for the correct file name.

\noindent {\tt IMAGE\_OUT} -- While DoPHOT is reducing an image, the
program continuously 
updates a version of the original picture from which all known objects
have been subtracted.  This subtracted picture can be saved when
DoPHOT is finished in the file specified by {\tt IMAGE\_OUT}.
If the string assigned to this keyword is blank,
the program assumes you don't wish to save the subtracted 
picture and none is written to disk.

\noindent {\tt EMP\_SUBRAS\_OUT} -- File name for writing out the adopted 
empirical PSF subraster and saving it when DoPHOT is finished.
If left blank, the empirical PSF image is not saved.

\noindent {\tt OBJECTS\_IN} -- This keyword specifies the name of
the file with the input object list.  This file is mandatory 
for a WARMSTART (see below) and will be requested if needed. 
If no WARMSTART is desired, this keyword is ignored.

\noindent {\tt OBJECTS\_OUT} -- This keyword identifies the name
of the output photometry file.  Since you have gone to the
trouble of using DoPHOT, you
presumably want to keep the output photometry; thus, if no legal file is 
specified here, the program will request its name until you
give it something it likes. 

\noindent {\tt SHADOWFILE\_IN} -- One option in DoPHOT is to 
do a WARMSTART (see below) and continue a reduction that 
failed due to an error in DoPHOT or a computer crash or
any similar problem.  The shadow files contain the intermediate
steps of the reductions, and if you have one you save a bit
of time on a WARMSTART.  If you are doing a WARMSTART, DoPHOT
will use this keyword to determine the name of the input
shadow file.  If this line is blank, the program assumes you 
don't have a shadow file (which is fine) even if the READSHD
flag is set to YES (see below).   If the program
finds a file name but can't open the file, it will ask for 
a name.  Shadow files are only stored in {\tt INTERNAL} format
(see {\tt OBJTYPE\_IN} and {\tt OBJTYPE\_OUT}, below).

\noindent {\tt SHADOWFILE\_OUT} --  If you plan to do a warm
start in the future or believe it would be prudent to have one
in case of future problems, you can save a shadow file.  The name
of such a file 
is specified using this keyword.  If no name is given, then
the program assumes you don't want an output shadow file.

\noindent {\tt ERRORS\_OUT} -- {\it new feature} If you save a 
shadow file, an identical file containing the errors on each of the 
shadow parameters is automatically generated.  The name of such 
a file is specified using this keyword.  If no name is given, the default 
file `error\_out' is created.

\noindent {\tt LOGFILE, LOGVERBOSITY} -- The user can specify
varying levels of detail in the messages that 
DoPHOT provides.  These messages range from information about
which threshold the program is currently working on, to 
iteration counters for the nonlinear least-squares fitting routine.
{\tt LOGFILE} specifies the name of the file into which these
messages are printed.  Any name that your operating system is
happy with is valid except one.  If this keyword is set 
equal to  TERM (all upper-case letters!), then DoPHOT's output
is directed to the terminal.  {\tt LOGVERBOSITY} specifies the
detail in the messages:  0 means no messages are printed out at
all and no file is opened even if {\tt LOGFILE} is specified.
Increasingly larger values of {\tt LOGVERBOSITY} (up to 4 is 
currently supported) provide progressively
more detailed messages.  Try different values to see what you 
like, but be warned that at full verbosity (4) 
you will get {\it many} messages, so
make sure you have loads of disk space.  Also, the program
(naturally) runs significantly slower if it is printing out
many messages.  The messages {\it are} useful, though, if 
things are going wrong. At low verbosity, DoPHOT's messages
help to reassure the user that the program is at work and provide
information how far along the reductions have proceeded.

\centerline{A.4.2  Infrequently updated parameters.}

These parameters are not typically changed very frequently
but may vary for images obtained with different 
detectors, for example.

\noindent {\tt RESIDNOISE} -- This parameter specifies how
much {\it extra} noise to add(subtract) to the noise file when
subtracting(adding) an object to an image and is 
designed to discourage DoPHOT from finding `phantom' objects
in the wings of previously-subtracted bright objects.

\noindent  {\tt EMP\_RESIDNOISE} -- If empirical PSFs are used, 
the images of stars should subtract extremely well, so the 
{\it extra} noise needed (see RESIDNOISE above) to prevent 
detection of `phantom' objects is much smaller. This is implemented 
via this parameter, which is used in lieu of RESIDNOISE when empirical 
PSFs are being fitted.

\noindent {\tt FOOTPRINT\_NOISE} --  The linear dimensions of a
star's footprint in the
noise file are enlarged by this amount.  The purpose of this
parameter is -- as in the case of {\tt RESIDNOISE} described
above -- to discourage DoPHOT from trying to find faint stars
in the residuals of the wings of bright or extended objects.
This linear expansion is {\bf not} performed when and empirical,
rather than analytic PSF is subtracted.

\noindent {\tt NPHSUB} -- When subtracting an object from an
image, DoPHOT goes out to where the 
PSF fitted to an object reaches this surface brightness.

\noindent {\tt NPHOB} -- When obliterating objects,
DoPHOT determines where the PSF that has been fitted
to an `obliteratable' object (see the discussion of {\tt ICRIT}, 
{\tt CENTINTMAX} and {\tt CTPERSAT} below) reaches a 
surface brightness equal to {\tt NPHOB}, in DN.  
The size of the side of an obliteration box for a given object is then set 
equal to
twice this linear distance.  The sizes of the $x$ and $y$ sides of 
obliteration boxes can differ if the PSF that has been fitted to the 
object is significantly elongated.  The calculation of the
obliteration box using {\tt NPHOB} does {\it not}
use the {\tt BETA4} and {\tt BETA6} parameters
described below.

\noindent {\tt ICRIT} -- If there are this many contiguous
saturated pixels (as determined by testing the pixel
values with {\tt ITOP}), then DoPHOT obliterates the object.
The size of the obliteration box is controlled by {\tt NPHOB}
(see above).  

\noindent {\tt CENTINTMAX} -- If the fitted central intensity of
an object exceeds this value, it is obliterated using {\tt NPHOB}
to determine the obliteration box size.  

\noindent {\tt CTPERSAT} -- If an object is to be obliterated
-- either because it contains too many bright pixels (the
{\tt ICRIT} test) or is simply too bright (the {\tt CENTINTMAX}
test) -- then its central intensity is assumed to be 
{\tt CTPERSAT} times the number of saturated pixels in the
object.  This helps DoPHOT calculate reasonably-sized
obliteration boxes.  

\noindent {\tt STARCOSKNOB} -- Every object found by DoPHOT
is initially fit as a single star and a cosmic ray.
If {\tt STARCOSKNOB} times $\chi^2$(cosmic ray) $\geq$ 
$\chi^2$(single star), the object is classified as a
cosmic ray.  

\noindent {\tt STARGALKNOB} -- Every non-cosmic ray
object is fitted as a galaxy after being fitted to a single star.
If the galaxy fit is significantly better than the fit for a
single star, then the object is also fit as a double star 
If {\tt STARGALKNOB} times $\chi^2$(galaxy) $\leq$ 
$\chi^2$(double star), the object is classified as a 
galaxy; otherwise as a double star.  

\noindent {\tt SNLIM7} -- An object's S/N must exceed this limit
to fit all seven parameters.  

\noindent {\tt SNLIM} -- A given pixel's S/N must exceed this limit
in order for it to be included in a fit subraster.

\noindent {\tt SNLIMMASK} -- This parameter is the minimum S/N that
must be detected through the mask in order for DoPHOT to 
trigger on an object.   
Excessively low values of
{\tt SNLIMMASK} will cause DoPHOT to trigger on noise as it
approaches the lower thresholds.  This increases execution time with 
very little gain in detection efficiency.

\noindent {\tt SNLIMCOS} -- This is the minimum S/N required for 
an object to be classified as a cosmic ray.  Without
this parameter, noise spikes (lots of them!) would be called cosmic rays at 
low thresholds.  

\noindent {\tt NBADLEFT, NBADRIGHT, NBADTOP, NBADBOT} -- Pixels
that are located closer to the edges of the image than these 
values are ignored
by DoPHOT -- sort of like `software trimming'.  Left, right,
top, and bottom correspond to small $x$, large $x$, large $y$, and
small $y$, respectively.

\noindent {\tt PSFTYPE} -- This flag specifies the form of the
analytic PSF used in DoPHOT's fits for objects.  The only
allowed response currently is PGAUSS, so no matter what you
specify in the parameter file, you'll get PGAUSS.  Additional 
PSF types may be available in future versions of DoPHOT.

\noindent {\tt SKYTYPE} -- This flag identifies the form of the
analytic sky function used by DoPHOT for finding objects.  
It is worth emphasizing that the individual sky values for 
fitted objects {\it are not} derived from this sky function --
the sky function 
is only used to help DoPHOT finding things on an image.  The
currently allowed options for this flag are PLANE, HUBBLE, and MEDIAN.
For a description of the latter, see Section 11.3.

\noindent {\tt JHXWID, JHYWID} --  These specify the HALF size of the median
filtering box along X and Y respectively. They must be integer values.
If you specify a non-positive value, the size will be scaled
automatically from the FWHM parameter.
The default for both is 0, which forces auto-scaling.

\noindent {\tt MPREC} --  This controls the precision to which 
the median model is
calculated. If the noise in the image background is say 100 DN, you do 
not need a model computed to 1 DN precision. 
You can ask instead for precision to $n$ DN by setting the value of the
MPREC parameter to $n$. This must be an integer. 
Non-positive values of MPREC will result in 
setting MPREC to 0.25*THRESHMIN.
The default is 1, the most conservative setting.

\noindent{\tt NTHPIX} -- This is really not a filtering parameter, 
but one used in 
the search for objects. It controls how often the search routine updates
the value of the background from the model (be it PLANE, HUBBLE or
MEDIAN) when initially raster searching along X for peaks 
in the data. NTHPIX is the 
interval in pixels for this update. Setting NTHPIX 0 or negative (the
default) sets NTHPIX = SQRT(X-size of image in pixels) to the nearest
integer. This is adequate for all but the grottiest backgrounds.
If you must have a smaller value you can change this to about 2*JHXWID.
Smaller values cause the search routine to run considerably slower.

\noindent {\tt OBJTYPE\_IN, OBJTYPE\_OUT} -- Three input/output formats are
supported by DoPHOT.  These are identified by the character
strings COMPLETE, INTERNAL, and INCOMPLETE.  The `COMPLETE' input/output
format type provides all the information DoPHOT has to offer about the
objects it has identified in an image using `friendly' units ({\it i.e.},
magnitudes, FWHM, {\it etc.}; Appendix B describes DoPHOT's data formats
in detail).  The 
`INTERNAL' input/output type provides the same information that is
provided in the `COMPLETE' format type but using the 
units employed internally by DoPHOT.
The `INCOMPLETE' output type is
compatible with DAOPHOT standard output files, but some information 
is lost if this format is chosen.  

The {\tt OBJTYPE\_IN}
flag identifies the input object file format type, and is needed if one
will be doing a ``warmstart''.
Only the COMPLETE, and INTERNAL format
types are allowed because there is not enough information in 
the  `INCOMPLETE' format files to initiate a warmstart.  The
{\tt OBJTYPE\_OUT} flag identifies the output file format type, and
all three options are allowed.  Note that shadow files 
are only written and read using `INTERNAL' format types
since these files are rarely of interest to most users.

\noindent {\tt PSDIR, PSNAME\_ROOT, CPSNAME\_ROOT, 
MPSNAME\_ROOT, RPSNAME\_ROOT} -- {\it new feature} If you want to 
output a postage stamp (PS) image for each obj detected (i.e. an image 
which is cut around the obj center and is the size of the model subtracted) 
specify a keyword for PSNAME\_ROOT such that the desired PS file names 
are `PSDIR/d\#\_PSNAME\_ROOT.fits' 
where the number corresponds to the number of the object in the object 
file.  Similar PS files can be created for the neighbor cleaned images of 
the objects, images of the models for each object, and images of just the 
residuals for each object once the neighbors and models have been 
subtracted off, by 
specifying names for CPSNAME\_ROOT, MPSNAME\_ROOT, or 
RPSNAME\_ROOT respectively.  Specifying a postage stamp directory 
PSDIR will not cause any files to be generated, but will properly direct files 
if any of the PSNAME\_ROOT keywords are specified. Postage stamps are 
a new feature DoPHOT 4.0 which are a useful diagnostic for seeing the 
quality of neighbor subtraction and the fit of any individual object, but 
generating PS images does slow down the run as individual images are 
generated for each object detected.

\centerline{A.4.3  Very rarely updated parameters.}

Not only are these parameters rarely changed, but they can cause
serious problems if adjusted indiscriminately.  Nevertheless,
they are included in the parameter file to increase DoPHOT's
flexibility in dealing with unforeseen reduction problems.  
However, even more so than for the parameters described above,
{\it be sure you understand the functions of these parameters
before changing their values.}

\noindent {\tt NFITITER} -- The value of this parameter controls
the maximum number of iterations DoPHOT performs during its
non-linear least-squares fits to individual objects on an
image.

\noindent {\tt NPARAM} -- NOW OBSOLETE- For developers, set to update 
in tuneup.c automatically for an updated model type.  For users, if you do 
not change the model, or if you use a model supplies with the package, this 
number will set correctly (to 7 for a pseudogaussian) automatically. 
Previously, NPARAM specified the maximum 
number of fit-parameters in the analytic PSF function.  For example, the 
fit-parameters used by the pseudo-gaussian function (which is specified by 
setting {\tt PSFTYPE} = PGAUSS; see above) are the sky, the $(x,y)$ centroid 
of the object, the central intensity, width parameters in $x$ and $y$ and a
cross term to account for tilted images.

\noindent {\tt NFITMAG} -- The maximum number of fit-parameters used
in fitting the analytic PSF to objects when the shape fit-parameters
are assumed.  Thus for the pseudo-gaussian function (see
{\tt PSFTYPE} above) these fit-parameters are the sky, the $(x,y)$ centroid
of the object, and the central intensity.  The shape fit-parameters 
are assumed to be equal to the current best guess that DoPHOT has
for these.

\noindent {\tt NFITSHAPE} -- NOW OBSOLETE- For developers, set to update 
in tuneup.c automatically for an updated model type.  For users, if you do 
not change the model, or if you use a model supplies with the package, this 
number will set correctly (to 7 for a pseudogaussian) automatically. 
Previously, NFITSHAPE specified the maximum number of fit-parameters used
in fitting the analytic PSF to objects when all the fit-parameters are
free to vary.  For the pseudo-gaussian function (see the description
for {\tt PSFTYPE} above) these fit-parameters are the same as 
described for {\tt NPARAM} above.

\noindent {\tt NFITBOXFIRST\_X, NFITBOXFIRST\_Y} --  Because the initial estimates
of the shape parameters provided by the user may not be especially good
(nor must they be), DoPHOT
should use a particularly large fitting-box on the first threshold in which stars are
detected.  This parameter controls the size of sides of the `first-pass' 
fit subraster boxes (see {\tt NFITBOX\_X} and {\tt NFITBOX\_Y} for 
related information).

\noindent {\tt N\_EMP\_SKIP} -- Setting this to a positive value N 
(0 is default) will tell DoPHOT to ignore the N brightest bonafide 
stars when selecting a star to use as the empirical PSF.

\noindent {\tt EMP\_STAR\_X, EMP\_STAR\_Y, EMP\_STAR\_Z} -- You can specify 
a particular star in the data frame as the empirical PSF template
by specifying its location in X and Y, and its approximate peak height
(Z) through these parameters. The default values are zero, which 
makes DoPHOT select a star itself, constrained by N\_EMP\_SKIP.

\noindent {\tt EMP\_PSF\_BOX} -- is the size of the empirical PSF subraster

\noindent {\tt EMP\_REJ\_RAD\_SIG} -- Used when selecting a PSF star. If a
bad pixel is encountered within V*sigma of the center of a star, it 
disqualifies it from being used as a template for the empirical PSF.
(For this purpose, sigma is FWHM/2.35).

\noindent {\tt CHI2MINBIG} -- This parameter specifies the critical
value of $\chi^2$ for three degrees of freedom
above which an object is classified as
``VERY BIG'' ({\it i.e.}, either a double star or a galaxy).  
Such objects are then subjected to fits as
double stars, and DoPHOT decides whether they are best classified
as galaxies or double stars. A related parameter is {\tt STARGALKNOB}
described above.

\noindent {\tt XTRA} -- This parameter has the same function as
{\tt CHI2MINBIG} except that it is used if the object has
previously already been classified as a member of a double star.

\noindent {\tt SIGMA1, SIGMA2, SIGMA3} -- These parameters control
the minimum fractional amount by which the PSF shape parameters
of a galaxy must vary from the shape parameters of a single
stellar object, and are used by DoPHOT to calculate the 
$\chi^2$ statistic for galaxy objects.

\noindent {\tt ENUFF4, ENUFF7} -- At least {\tt ENUFF4} of the 
pixels in a fit subraster must be present in order to 
perform a four-parameter fit (see {\tt NFITMAG} above).
{\tt ENUFF7} specifies this limit for seven-parameter
fits (see {\tt NFITSHAPE} above).

\noindent {\tt COSOBLSIZE} -- This specifies the size of the
obliteration box (in both $x$ and $y$ directions) that is
employed to remove objects that have been classified as
cosmic rays.

\noindent {\tt APMAG\_MAXERR} -- Only aperture photometry 
results with errors smaller than the value specified
by this parameter are reported in the output photometry
file.

\noindent {\tt PIXTHRESH} -- This parameter allows DoPHOT's
finding algorithm to trigger on pixels that are above
{\tt PIXTHRESH} times the local noise.  This parameter
represents one of the masking tape/baling wire type fixes
that has been incorporated into the code in order to 
allow the finding algorithm to find faint stars in 
well-sampled data.  By setting {\tt PIXTHRESH} too low,
however, you force the program waste time trying to
find objects where only a single noisy pixel exists.  
Binning well-sampled data and leaving {\tt PIXTHRESH}
at 1.0 is usually more effective at finding faint stars.

\noindent {\tt BETA4, BETA6} -- These parameters modify the
$z^4$ and $z^6$ terms in the pseudo-gaussian PSF function
as illustrated in Equation (1) of Section 7.2.  If you wish
to experiment with these parameters, you can  
plot Equation (1) with different values of
{\tt BETA4} and {\tt BETA6} to appreciate how they 
affect the shape of the pseudo-gaussian PSF function used
in the present version of DoPHOT and to compare the
analytic PSF with the radial profiles of actual PSFs.  For most
real data, the pseudo-gaussian will never perfectly fit
the observed PSF, but you will probably be able to find
values of {\tt BETA4} and {\tt BETA6} that do a good job
far out into the wings of the observed PSF.  We suggest letting
{\tt BETA4} = {\tt BETA6} while varying them for the first time
on new data sets.

\noindent D. Fit-parameter limits.  

The fit-parameter limits below provide convergence limits for
DoPHOT's non-linear least squares fitting routine and help avoid the 
program from producing absurd results or attempting to drive
the fit-parameters to values that result in numeric overflows.  Two sets of 
seven fit-parameter limits are used by DoPHOT and are identified
with indices running from 1 to 7.    For reference, these indices represent 
(1) sky, (2) $x$-position of center, (3) $y$-position of the center,
(4) central intensity above sky, (5) pseudo-gaussian sigma in the 
$x$-direction, (6) the pseudo-gaussian cross term, and (7) the pseudo-gaussian
sigma in the $y$-direction, respectively.  These fit-parameters
are defined by Equation (1) in Section 7.2.

\noindent {\tt RELACC1} through {\tt RELACC11} -- These parameters control
the convergence criteria for the fit-parameters.  Positive values for 
any {\tt RELACC} parameter refer to fractional changes in the
fit-parameters between successive iterations, 
while negative values refer to absolute changes.
For example, if {\tt RELACC1} = 0.01, then if the sky value 
changes by less than 1\%\ between successive iterations, that parameter
will be flagged as having converged.  Similarly, if
{\tt RELACC3} = $-0.03$, then if the object's $y$ position  changes
by less than 0.03 pixels between iterations, that parameter is flagged as
having converged.    It should be noted, however, that DoPHOT
continues to iterate until {\it all} of of the fit-parameters have
satisfied their individual convergence criteria.  Because of this, the
actual convergence achieved for most parameters is much better than
one might guess from the values of the {\tt RELACC} parameters.  Note that for
the standard pseudo gaussian model, only {\tt RELACC1-7} are used. 
{\tt RELACC8-11} are left available for models with additional parameters.


\noindent {\tt ABSLIM1} through {\tt ABSLIM7} -- These parameters
control the permitted range of the fit-parameters and are 
used between iterations of the non-linear least-squares PSF fit.  
Positive
values for any {\tt ABSLIM} parameter controls fractional changes
in the fit-parameters between sucessive iterations.
Negative values for 
the {\tt ABSLIM} parameters restrict the absolute
ranges of the fit-parameters.  Thus, setting {\tt ABSLIM1} = 100.0 causes
DoPHOT to halt its attempt to fit the current object if the sky value
changes by more than a factor of 100 between iterations.  
Setting {\tt ABSLIM4} = $-1.0 \times 10^8$
causes a fit to halt if the central intensity of an object ever
exceeds $10 \times 10^8$ DN.

\centerline{\bf A.5  A Sample DoPHOT 4.0 Parameter File}

\vskip1em

{\obeylines \baselineskip=12pt \parindent=0pt
{\tt
$=$  Default Parameter File for DoPHOT Version 4.0 .      June 2012; MIT.
$=$
$=$  THIS IS THE MASTER DoPHOT DEFAULT PARAMETER FILE AND IT SHOULD NOT BE
$=$  DELETED OR MODIFIED!
$=$
$=$  DoPHOT Parameter files are ALMOST pseudo-FITS headers.
$=$  Each line for which a keyword is assigned a value must have an
$=$  equals sign and at least one blank space immediately before and after it.
$=$  Comments may be placed after the parameter value on any line and the equal
$=$  sign may be located anywhere on a line.  A maximum of 80 characters is
$=$  allowed per line.  Character variables MUST be enclosed in single quotes.
$=$  All keyword names must start in the first column.

$=$  As a hold over from when this was in Fortran, normal Fortran naming
$=$  conventions are used to identify integer and real variables:  keywords
$=$  starting with letters between I and N (inclusive) represent integer
$=$  variables, all others are reals except for variables in quotes which are
$=$  assumed to be character variables.  No restrictions apply to character
$=$  variable keyword names except that all keyword names must be kept to under
$=$  20 characters.

$=$  Null character variables are represented as having
$=$  no equals sign.  An equals sign means a value is assigned. This is
$=$  DIFFERENT THAN THE FORTRAN VERSION!!!
$=$
$=$  In this master file, the parameters are grouped according to the typical
$=$  frequency with which they are modified.  Short descriptions for each
$=$  parameter are provided following the default value.   These defaults are
$=$  designed to work for a `typical' TI CCD image.
$=$
$=$
$=$  FREQUENTLY UPDATED PARAMETERS.  These parameters mostly depend on the
$=$  typical size of stellar objects on a given frame and on the mean sky value
$=$  for the image.  These parameters tend to be different for every frame.
$=$  Also included are the bookkeeping parameters that specify file names.
$=$
FWHM $=$ 2.0             Approx FWHM of objects (pixels) along major axis.
AXIS\_RATIO $=$ 1.0       For star objects.  AR$=$b/a; b$=$minor axis.
TILT $=$ 0.0             Angle of major axis in degrees; +x$=$0; +y$=$90.
SKY $=$ 10.0             Approximate mean sky value in data numbers.
NFITBOX\_X $=$ 7          Size of fit box in the x-direction.
NFITBOX\_Y $=$ 7          Size of fit box in the y-direction.
MASKBOX\_X $=$ 5          Size of mask box size in x.
MASKBOX\_Y $=$ 5          Size of mask box size in y.
APBOX\_X $=$ 13.0         Size of aperture photometry box in x.
APBOX\_Y $=$ 13.0         Size of aperture photometry box in y.
IBOTTOM $=$ -50          Lowest allowed data value in data numbers.
ITOP $=$ 16384           Maximum allowed data value in data numbers.
THRESHMIN $=$ 20.0       Value of lowest threshold.
THRESHMAX $=$ 10000.0    Value of maximum threshold.
THRESHEMP $=$ 0.0        Fit empirical PSF at and below this value.
THRESHDEC $=$ 1.0        Threshold decrement in powers-of-2.
MAX\_SOUGHT $=$ 32768     Quit after this number of improved stars.
MAX\_PERF  $=$ 2000       Only average up to this number of stars to get shape parameters.
RANGE\_MAG $=$ 30.        Sets threshhold some magnitudes fainter than brightest fit.
EPERDN $=$ 2.0           Electrons per data number.
RDNOISE $=$ 15.0         Readout noise in$=$
AUTOSCALE $=$ 'NO'       Auto-scaling of sizes by FWHM.
AUTOTHRESH $=$ 'NO'      Auto-scaling of thresholds.
FIXPOS $=$ 'NO'          Fix star positions?
$=$
PARAMS\_DEFAULT $=$ 'param\_default\_c'  Default parameters file name.
PARAMS\_OUT $=$ 'param\_out'             Output parameters file name.
IMAGE\_IN $=$ 'image\_in'                Input image name.
IMAGE\_OUT $=$ 'image\_out'              Output image name.
EMP\_SUBRAS\_OUT                       Empirical PSF subraster (most recent).
OBJECTS\_IN                           Input object list file name.
OBJECTS\_OUT $=$ 'objects\_out'          Output object list file name.
SHADOWFILE\_IN                        Input shadow file name.
SHADOWFILE\_OUT $=$ 'shadowfile\_out'    Output shadow file name.
ERRORS\_OUT $=$ 'error\_out'  Errors on fit to be output if shadow file is requested
LOGFILE $=$ 'TERM'                     Log file name.  TERM for screen.
LOGVERBOSITY $=$ 1                     Verbosity of log file; (0-4).
$=$
$=$
$=$  OCCASIONALLY UPDATED PARAMETERS.  These parameters tend to not change for a
$=$  set of images obtained during a single observing run or for frames of a
$=$  single field.  The defaults for the flags PSFTYPE, SKYTYPE, and OUTTYPE are
$=$  PGAUSS, PLANE, and FULL, respectively.
$=$
RESIDNOISE $=$ 0.3       Fraction of noise to ADD to noise file (analytic PSF).
EMP\_RESIDNOISE $=$ 0.03  Fraction of noise to ADD to noise file (empirical PSF).
FOOTPRINT\_NOISE $=$ 1.3  Expand stars in noise file by this amount (analytic).
NPHSUB $=$ 1             Limiting surface brightness for subtractions.
NPHOB $=$ 1              Limiting surface brightness for obliterations.
ICRIT $=$ 10             Obliterate if \# of pixels > ITOP exceeds this.
CENTINTMAX $=$ 20000.0   Obliterate if central intensity exceeds this.
CTPERSAT $=$ 1.0e4       Assumed intensity for saturated pixels.
STARGALKNOB $=$ 1.0      Star/galaxy discriminator.
STARCOSKNOB $=$ 1.0      Object/cosmic-ray discriminator.
SNLIM7 $=$ 7.0           Minimum S/N for 7-parameter fit.
SNLIM $=$ 0.5            Minimum S/N for a pixel to be in fit subraster.
SNLIMMASK $=$ 4.0        Minimum S/N through mask to identify an object.
SNLIMCOS  $=$ 3.0        Minimum S/N to be called a cosmic ray.
NBADLEFT $=$ 0           Ignore pixels closer to the left edge than this.
NBADRIGHT $=$ 0          Ignore pixels closer to the right edge than this.
NBADTOP $=$ 0            Ignore pixels closer to the top edge than this.
NBADBOT $=$ 0            Ignore pixels closer to the bottom edge than this.
$=$
PSFTYPE $=$ 'PGAUSS'        PSF type: (PGAUSS, GAUSS, EXTPGAUSS)
SKYTYPE $=$ 'PLANE'         SKY type: (PLANE, HUBBLE, MEDIAN)
$=$
JHXWID $=$ 0              X Half-size of median box (.le. 0 -> autoscale)
JHYWID $=$ 0              Y (same as above)
MPREC $=$ 1               Median precision in DN (use .le. 0 for autocalc)
NTHPIX $=$ 0              Frequency of sky updates in pixels for 1st pass
$=$
$=$ Default value of NTHPIX is zero. DoPHOT translates it to SQRT(NFAST)
$=$ once the data frame is read. For MEDIAN sky, positive NTHPIX gives
$=$ control over NTHPIX for finer initial sky sample.
$=$
$=$
OBJTYPE\_IN $=$ 'COMPLETE'   Input format: (COMPLETE, INTERNAL)
OBJTYPE\_OUT $=$ 'COMPLETE'  Output format: (COMPLETE, INCOMPLETE, INTERNAL)
$=$
PSDIR $=$ './'               directory in which to put the PS files
PSNAME\_ROOT                ps files names: 'd\#\#\#\#\_PSNAME\_ROOT.fits'
CPSNAME\_ROOT               cleaned  ps file names: 'd\#\#\#\#\_CPSNAME\_ROOT.fits'
MPSNAME\_ROOT               model    ps file names: 'd\#\#\#\#\_MPSNAME\_ROOT.fits'
RPSNAME\_ROOT               residual ps file names: 'd\#\#\#\#\_RPSNAME\_ROOT.fits'
$=$
$=$  RARELY UPDATED PARAMETERS.  These are specialized parameters that rarely
$=$  need changing even when measuring images of very different sorts of fields
$=$  and/or from different telescope/detector combinations.
$=$
NFITITER $=$ 10          Maximum number of iterations.
$=$NPARAM $=$ 7            OBSOLETE Maximum number of PSF fit parameters. obsolete, now set with model
NFITMAG $=$ 4            No. of PSF parameters to get magnitudes.
$=$NFITSHAPE $=$ 7         OBSOLETE No. of PSF parameters to get shape and mags.  obsolete
NFITBOXFIRST\_X $=$ 31    Size of fit box in x for first pass.
NFITBOXFIRST\_Y $=$ 31    Size of fit box in y for first pass.
N\_EMP\_SKIP $=$ 0         Skip n stars brighter than empirical template.
EMP\_STAR\_X $=$ 0         X position of empirical template.
EMP\_STAR\_Y $=$ 0         Y position of empirical template.
EMP\_STAR\_Z $=$ 0         Central intensity of empirical template.
NEMP\_PSF\_BOX $=$ 33      Size of empirical PSF box
EMP\_REJ\_RAD\_SIG $=$ 2.0  Minimum distance of bad pixel from peak in sigma.
CHI2MINBIG $=$ 16        Critical CHI-squared for a large object.
XTRA $=$ 25              We need more S/N if some pixels are missing.
SIGMA1 $=$ 0.10          Max. frac. scatter in sigma\_x for stars.
SIGMA2 $=$ 0.10          Max. scatter in xy cross term for stars.
SIGMA3 $=$ 0.10          Max. frac. scatter in sigma\_y for stars.
ENUFF4 $=$ 0.50          Fraction of pixels needed for 4-param fit.
ENUFF7 $=$ 0.65          Fraction of pixels needed for 7-param fit.
COSOBLSIZE $=$ 0.9       Size of obliteration box for a cosmic ray.
APMAG\_MAXERR $=$ 0.1     Max anticipated error for aperture phot report.
PIXTHRESH $=$ 1.0        Trigger on pixels higher than noise*PIXTHRESH.
BETA4 $=$ 1.0            R**4 coefficient modifier.
BETA6 $=$ 1.0            R**6 coefficient modifier.
$=$
$=$
$=$  AUTO SCALING PARAMETERS.  These parameters are used only if the auto-scaling
$=$  flags are turned on.  Box sizes, and threshold levels can be scaled
$=$  according to the FWHM of objects and the sky and readout noise values.
$=$
SCALEFITBOX $=$ 3.0      Size of fit box in units of FWHM.
FITBOXMIN $=$ 5.0        Smallest allowed fit box size.
SCALEAPBOX $=$ 6.0       Size of aperture phot box in units of FWHM.
APBOXMIN $=$ 7.0         Smallest allowed aperture phot box size.
SCALEMASKBOX $=$ 1.5     Size of mask box in units of FWHM.
AMASKBOXMIN $=$ 5.0      Smallest allowed mask box size.
SIGMAIBOTTOM $=$ 10.0    Level of IBOTTOM below sky in units of noise.
SIGMATHRESHMIN $=$ 2.0   Level of THRESHMIN above sky in units of noise.
$=$
$=$
$=$  PARAMETER LIMITS.  These variables limit the legal ranges of the PSF and
$=$  sky parameters.  Be sure to understand their function well before changing
$=$  any of these values.  Positive values refer to fractional changes; negative
$=$  values to absolute changes; zero ABSLIM's turn the corresponding tests off.
$=$
RELACC1 $=$ 0.01         Convergence criterion for sky.
RELACC2 $=$ -0.03        Convergence criterion for x-position.
RELACC3 $=$ -0.03        Convergence criterion for y-position.
RELACC4 $=$ 0.01         Convergence criterion for for central intensity.
RELACC5 $=$ 0.03         Convergence criterion for sigma-x.
RELACC6 $=$ 0.1          Convergence criterion for sigma-xy.
RELACC7 $=$ 0.03         Convergence criterion for sigma-y.
RELACC8 $=$ 0.03         Convergence criterion for additional parameter 1
RELACC9 $=$ 0.03         Convergence criterion for additional parameter 2
RELACC10 $=$ 0.03        Convergence criterion for additional parameter 3
RELACC11 $=$ 0.03        Convergence criterion for additional parameter 4
$=$
ABSLIM1 $=$ -1.0e8       Allowed range for sky value.
ABSLIM2 $=$ -1.0e3       Allowed range for x-position.
ABSLIM3 $=$ -1.0e3       Allowed range for y-position.
ABSLIM4 $=$ -1.0e8       Allowed range for central intensity.
ABSLIM5 $=$ -1.0e3       Allowed range for sigma-x.
ABSLIM6 $=$ -1.0e3       Allowed range for sigma-xy.
ABSLIM7 $=$ -1.0e3       Allowed range for sigma-y.
END
}
}
\vfill
\eject