\centerline{\bf 9. MONITORING OUTPUT AND DIAGNOSING DIFFICULTIES}

\centerline{\bf 9.1  Monitoring Output}

The single most useful thing one can do to get a qualitative
feel for how DoPHOT is doing is to create a synthetic image
by subtracting the object-subtracted image from the original
image.  Displaying this will show whether DoPHOT is
successfully classifying galaxies and double-stars, and
whether spurious objects are being detected.

Another useful check is to compare outputs obtained for two
separate images of the same field.  One might ask how well
the fit magnitudes agree, after adjustment for possible
variations in transparency and correction to an aperture
magnitude system, how well the classifications and positions
agree, and which objects were detected on one frame but not
on the other.  Comparisons such as these can profitably be
done by hand, but can also be done using matching programs
which may be available from the person in the office next
door.

Much can be learned just from examination of the output
object list.  The scatter in the difference between
aperture and fit magnitudes should be consistent with the
quoted errors.   The
object classifications should make sense ({\it e.g.}, in an open
cluster most of the objects should be single stars and not
galaxies and split stars, while in a more crowded globular cluster
field, one might find a number of double stars, and in a high-latitude
field galaxies should be common).  The uncertainties
in the fit magnitudes should be small for bright stars
and larger for faint stars.  They will be larger for
stars which lie near other stars.

One can produce a large ``log'' file records DoPHOT's
progress in identifying objects, estimating positions and
magnitudes and shapes, classifying objects, and determining
typical shape parameters and sky values.  The output of the 
``log'' file can be easily directed to the user's terminal or
window.  Since the entries in this file
are more or less self explanatory, no attempt will be made
here to describe the (many) possible entries in the log file.
One useful quantity reported in the log file is the number
of pixels used in fitting a model.  If this is very much
smaller than the number computed from the dimensions of the
fit subraster, it may be that the object is at the edge of
the image, or that a nearby object has been ``obliterated.''

The user would be well advised to look at the output
parameters file to make sure that the values used were those
intended.  This file also serves as a record of the
reduction of that image with one caveat described in Appendix
A.  With the output parameters file, the user can {\it
precisely} reproduce that reduction.

The user is also encouraged to reduce some image that has
been reduced with some other program, perhaps borrowing one
from a colleague who wouldn't mind having his reduction
double checked.
