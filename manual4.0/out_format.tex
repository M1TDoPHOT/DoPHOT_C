\centerline{\bf APPENDIX B:  OUTPUT FILE FORMATS}

Formats for object photometry output can be done in 3 different 
ways, INTERNAL, COMPLETE and INCOMPLETE. 
Warmstarts are possible only from the INTERNAL or COMPLETE types, 
and additional models will only support outputs in the INTERNAL and 
COMPLETE types as well.
It is important to note that the INTERNAL format measures $x$ and $y$
positions assuming that the center of the first pixel in a row or column
of the image is 1.0, whereas for the COMPLETE and INCOMPLETE formats, 
the centers of the first pixels are measured as 0.5, in agreement with 
most conventions. 

\vskip1em

\noindent INTERNAL format.

This is the form in which DoPHOT internally tracks the objects
and their associated properties. For each object the following 
quantities are reported (the last 5 are from empirical PSF fitting, 
the remainder from analytic fitting or aperture measures):

\item{1)} Object Number
\item{2)} Object type
\item{3)} Fit sky value
\item{4)} $x$ position
\item{5)} $y$ position
\item{6)} Height (maximum) of fitted model PSF in Data numbers
\item{7)} $\sigma_x^2$ (see section 7.2 for the form of fitted function)
\item{8)} $\sigma_{xy}$ (as above)
\item{9)} $\sigma_y^2$ (as above)
\item{9.5)}*optional extra parameters from alternate models
\item{10)} Aperture flux (in DN)
\item{11)} Aperture sky value
\item{12)} Aperture magnitude minus fit magnitude
\item{13)} Estimated error in fit magnitude
\item{14)} Extendedness parameter
\item{15} Sky value from empirical fit
\item{16} Height (maximum) of fitted empirical PSF in Data numbers
\item{17} $x$ position from empirical PSF fitting
\item{18} $y$ position from empirical PSF fitting
\item{19} Estimated error in fit magnitude from empirical PSF  

The format of the output is:

\noindent {\tt  \%4d \%2d \%10.4E \%10.2f \%10.2f \%10.3E \%10.3E \%10.3E 
\%10.3E \%10.3E \%10.3E \%7.3f \%7.3f \%10.2E \%10.3E \%10.3E \%10.2f \%10.2f \%7.3f}

\noindent For Object type 8 (obliterations), item 7 and 9 above are the 
$x$ and $y$ dimensions of the obliterated subraster. For a rectangular obliteration, 
a value of 0.0 is stored in item 8 above and items 7 and 9 are the full x and y widths
in pixels.  For elliptical obliterations, the tilt of the ellipse in degrees in stored in item
8 above and items 7 and 9 are the semi major and minor axis widths.  If an object is
 judged to be a cosmic-ray, in which case the value in item 8 is $-1.0$.  Thus, for
 obliterating an ellipse, do not set the value of the tilt to identically $-1.0$ or $0.0$, 
 though $-1.01$ or $0.01$ will do just fine.  For similar reasons discussed below,
 avoid exactly $90.0$ degrees as well.  All other values are fine.

For the extended pseudogaussian model, where $\beta_4$ and $\beta_6$ are free parameters of the fit, the $\beta_4$ and $\beta_6$ values as well as an integer flag for each object are placed between items 9 and 10.  The flag is 0 if the extended pseudogaussian model is indeed fit to the object in question and subtracted from the image.  The flag is 1 if the extended pseudogaussian model did not converge on the object and so a standard pseudogaussian model ($\beta_4$ and $\beta_6$ fixed to the values specified in the parameter files) was fit to the object.   For objects of type 2 which are flagged as fitted with a pseudogaussian model rather than an extended pseudogaussian model, this same pseudogaussian model is subtracted from the image, regardless the reported parameters in the $\beta_4$ and $\beta_6$ columns.  For objects of types 1 and 3 which converge with the pseudogaussian model but not the extended pseudogaussian model, a typical star model (extended pseudogaussian) whose shape parameters are composed of the weighted averages of the converged type 1 stars are subtracted from that position-- the same as for all other stars.  However the shadow and error file parameters will reflect the fit from the converged pseudogaussian model, and the $\beta_4$ and $\beta_6$ columns in those files should be ignored.

\vskip1em

\noindent COMPLETE format (also sometimes referred to as ``external'' format
in the manual).


 This is the form that is most convenient for the user.
For each object, the following quantities are put out (the last 5 
are from empirical PSF fitting, the remainder from analytic fitting 
or aperture measures):

\item{1)} Object Number
\item{2)} Object type
\item{3)} $x$ position
\item{4)} $y$ position
\item{5)} Fit magnitude
\item{6)} Estimated error in fit magnitude
\item{7)} Fit sky value
\item{8)} FWHM of major axis
\item{9)} FWHM of minor axis
\item{10)} Tilt (degrees: 0 implies major axis along positive $x$-axis;
90 is along positive $y$-axis)
\item{10.5)}*optional extra parameters from alternate models
\item{11)} Extendedness parameter
\item{12)} Aperture magnitude  
\item{13)} estimated error in aperture magnitude
\item{14)} aperture sky value
\item{15)} aperture mag minus fit mag
\item{16} Sky value from empirical fit
\item{17} Fit magnitude from empirical PSF 
\item{18} $x$ position from empirical PSF fitting
\item{19} $y$ position from empirical PSF fitting
\item{20} Estimated error in fit magnitude from empirical PSF  

The format of the output line is:

\noindent {\tt  \%4d \%2d \%8.2f \%8.2f \%8.3f \%6.3f \%9.2f \%9.3f \%9.3f \%7.2f 
\%10.3E \%10.3E \%10.3E \%10.3E \%10.2E \%8.3f \%6.3f \%9.2f \%7.3f \%9.2f 
\%8.3f \%8.2f \%8.2f \%6.3f }

If the object type is 8, the shape descriptors for rectangular obliterations are 
set as follows:

\noindent FWHM of 
major axis shows the longer side of the obliterated rectangle.

\noindent FWHM 
of minor axis shows the shorter side of the obliterated rectangle.

\noindent Tilt of 0.0 
shows that the long side of the obliterated rectangle is along $x$.

\noindent Tilt of 90. shows that the long side of the obliterated 
rectangle is along $y$.

All other tilt values indicate an ellipse was obliterated rather than a 
rectangle and the `FWHM' values are in fact the semi-major and minor 
widths of the obliterated ellipse.

\noindent Empirical fit magnitudes are flagged at 99.999 is the fitted 
intensity was zero or negative. It is flagged at -99.999 if no empirical 
PSF fit was attempted for that object.

\vskip1em

\noindent INCOMPLETE format.

For the die-hard DAOPHOT compatibility freak.

Output quantities in order are:

\item{1)} Object Number
\item{2)} $x$ position
\item{3)} $y$ position
\item{4)} Fit magnitude
\item{5)} Estimated error in fit magnitude
\item{6)} Fit sky value
\item{7)} Object type (converted to floating number)
\item{8)} Extendedness parameter
\item{9)} Aperture mag minus fit mag

The format of the output lines is:

\noindent {\tt "\%6d\%9.2f\%9.2f\%9.3f\%9.3f\%9.3f\%8d.\%9.2f\%9.3f}


