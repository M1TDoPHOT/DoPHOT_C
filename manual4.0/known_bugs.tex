\centerline{\bf 10. KNOWN BUGS AND SHORTCOMINGS}

\centerline{\bf 10.1 Maximum Number of Objects}

The maximum number of objects is compiled into DoPHOT through a
\#define in {\bf structs/tuneable.h} 
{\it The user must recompile and relink all} 
DoPHOT {\it modules after changing any \#define s} 
as described in Appendix A.  
At present DoPHOT does {\it not} check this
maximum before adding objects to the object list.

\centerline{\bf 10.2 Initial FWHM Limitations}

When DoPHOT encounters objects on its first pass through the
data, it must decide whether they are single stars, galaxies
or double-stars.  This decision is based on the user
supplied full width at half maximum.  If this number is too
small, DoPHOT will decide that {\it all} the single stars it
encounters are galaxies and double-stars.  Since it
encounters no single stars, it cannot update its notion of
what a typical star looks like and single stars are
classified as galaxies and double-stars on all subsequent
passes through the data as well.  Thus, the user's best estimate of
the FWHM is automatically multiplied by 1.2 to ensure that 
DoPHOT's initial guess of the shape parameters are somewhat 
overestimated.  Note that the FWHM provided by the user, 
{\it not} the inflated value, is saved in the output parameters file.

\centerline{\bf 10.3 Lost Siblings}

DoPHOT does not maintain pointers linking those objects
which are double-star siblings.  As a result, the values
obtained at the moment when the stars are split are probably
{\it better} than those ultimately reported.  When one
sibling of a pair is encountered on a subsequent pass
through the data, it is fit as a single star with the
``typical'' shape parameters.  Since it's sibling has been
subtracted, the noise in the nearby pixels is higher than it
would otherwise be, increasing the formal uncertainties from
the fit.  But since the sibling is {\it not} being fit
simultaneously, the error estimates for the parameters do
not take into account the covariance with the estimates for
the sibling.  This second effect would underestimate errors.

\centerline{\bf 10.4 No Longer and issue: Maximum Image Size}

We assume that all modern computers can dynamically allocate 
memory.  If your computer cannot, stick with DoPHOT 3.0 in Fortran.  
The image size is read form the image header and the memory for 
the image is dynamically allocated to the correct size.

\vskip3em

\centerline{\bf ACKNOWLEGEMENTS}

The authors gratefully acknowledge the help of colleagues
who generously agreed to try using early versions of DoPHOT,
tested it on a variety of different kinds data, reported the
results of their experiments, and suggested a great number
of the improvements which have been incorporated into the
present version: John Caldwell, Kem Cook, Eileen Friel,
Wendy Freedman, Doug Geisler and John Tonry.  They also
thank John Tonry for his routines to read ``disk fits''
files, even though they are no longer implemented.  
DoPHOT was originally written with the support of 
the  U.S. National Science Foundation through grant AST83-18504.
DoPHOT 4.0 modifications were supported by a NSF Graduate 
Research Fellowship to Rebecca Levinson.
