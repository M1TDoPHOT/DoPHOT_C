\centerline{\bf 4.  CONTENTS OF VERSION 4.0 AND HOW TO USE THEM}

	The program and data files provided to you have been arranged 
into a subdirectory structure, which you are strongly urged to maintain.
In versions 3.0 and prior, the program was arranged into a more 
redundant subdirectory structure with the underlying intent to maintain 
always a pristine ORIGINAL copy of the code as reference, and to have 
an easy facility to check that code that has been modified produces 
IDENTICAL results when run on the supplied test images with the 
supplied input data and parameters. We believe that with modern 
source control the old subdirectory structure may be cumbersome to 
users rather than helpful, so it has been dropped in 4.0.  We do still 
encourage that users keep and maintain a pristine copy of DoPHOT
at all times on their machines.  As with previous versions, all modified 
code should produce IDENTICAL results when run on the supplied test 
images with the supplied input data and parameters.  Whether the user 
chooses to use GIT and make their working copy a branch off of a pristine 
master or simply download DoPHOT twice, one copy for modifications 
one for verification is at his or her discretion.

\centerline{\bf 4.1 CFITSIO precursor}
	DoPHOT 4.0 requires the CFITSIO library which is not included 
with DoPHOT.  It can be obtained at 
http://heasarc.gsfc.nasa.gov/docs/software/fitsio/fitsio.html
and installed by following the guidelines on that page.  Once it is installed
(and working), edit {\bf Makefile} in the top level, MAIN directory
supplied with this package to reflect the path of the CFITSIO installation. 
Unfortunately, you will NEED TO MODIFY THE MAKEFILE, EVEN IN 
THE `PRISTINE' DoPHOT COPY as well as the working copy, however 
this is the ONLY FILE YOU SHOULD EVER MODIFY IN THE PRISTINE 
COPY. 

\centerline{\bf 4.2 UNIX Systems}

	The tar supplied to you, should, when properly read, 
produce a main-level directory (which can be a sub-directory in 
your general subdirectory tree), which will be referred to as MAIN.
MAIN contains 
\item{1)} {\bf README} containing general information and instruction file(s) that you 
should read before proceeding.
\item{2)} subdirectory {\bf manual} containing a TEXable version of this manual.
\item{3)} a {\bf Makefile}
\item{4)} the original Version4\_0 code as .c files.
\item{5)} subdirectory {\bf headers} containing headers for most of the source 
routines
\item{6)} subdirectory {\bf structs} containing additional header files with struct 
information
\item{7)} subdirectory {\bf verif\_data} containing test files
\item{8)} subdirectory {\bf working\_data} which is empty.

Run 'make test' to populate the {\bf verif\_data} subdirectory with a dophot 
executable.
Alternately, run 'make dophot' from MAIN to populate {\bf working\_data} 
with a dophot executable.

In either case, the {\bf working\_source} subdirectory will populate with
.o files and any errors in your modified Makefile or CFITSIO installation 
will now appear.

	On a pristine copy of DoPHOT version4\_0, the {\bf working\_data}
subdirectory is empty. By running the `make' described above, an executable
image of {\bf dophot} is created here. You may choose to use this 
as your work area for DoPHOT; we urge that you do. Before you can do so, 
you will need proper parameter files and other optional input files
(as described elsewhere in this manual) in addition to the image on which
you wish to run DoPHOT. 

	The {\bf verif\_data} subdirectory comes with the test image 
{\bf J0159.8.chip\_01.fits}, the default parameter file 
{\bf param\_default\_c}, a file containing the modified parameters {\bf pm}, 
and a list of input objects {\bf obj\_in\_J0159.8.chip\_01} for the field. Additionally 
{\bf verif\_data} contains a subdirectory {\bf compare\_out} containing files 
intended for verification that the supplied program or any modification you 
make is executing correctly.  By running 'make test' in 
{\bf working\_to\_compilable}, {\bf verif\_data} will be populated with a 
dophot executable.

Let us walk through and see how to run dophot on the galaxy cluster image 
called {\bf J0159.8.chip\_01.fits}.  
Type `./dophot' at the command line (or equivalent for your machine).  You 
should be prompted for a parameter modification file.  Type in 'pm', which 
contains the non-default parameters for this test file, and hit return.  'pm' also 
informs DoPHOT as to which file contains the default parameters, namely 
{\bf param\_default\_c}, which should never be modified.  

DoPHOT should now execute in a number of seconds and report 6 output files 
matching the ones in the {\bf compare\_out} subdirectory. Feel free to examine 
all of them, but in particular diff {\bf obj\_out\_J0159.8.chip\_01} with 
the {\bf compare\_out} file of the same name.  If there are any discrepancies, 
something has gone wrong or you are not working with a pristine copy of 
DoPHOT. 











