\centerline{\bf 7.  DETAILS FOR CONNOISSEURS}

This section is intended for primarily for users who have
run DoPHOT several times and who are comfortable with the
material covered in the previous two sections.

\centerline{\bf 7.1 Pixels versus Models}

A theme that keeps recurring in these pages is that of
modeling an array of pixels.  A small subraster is filtered
through a model profile to determine whether or not an
object should be added to the list.  A somewhat larger
subraster is fit with a model to determine the shape of an
image.  The image is modeled by the entire list of objects.
Wherever possible the sky, as sampled by the pixels, is
modeled.  A goal in the design of DoPHOT was to
use model fitting in preference to pixel based algorithms
wherever possible.

An immediate advantage of such a model based approach is
that one can fit the model even when the image is not
uniformly sampled.  If pixels are missing for some reason
(e.g. a cosmic ray, or a bad column, or a saturated pixel at
the center of an image), one can still fit a model to the
data.  Odd-shaped pixels are also easily accommodated.  If
one had data which was oversampled by a factor of 2, such
that the FWHM was of order 4 pixels, then it should be
possible (and sometimes desirable) 
to compress the data by a factor of 2 in both
directions and obtain very nearly the same output object
list.  

This ideal is sacrificed in the use of a single high pixel
to trigger the subsequent test for an object, in the use of
a single pixel model for a particle event, and (inevitably)
in the specification of raster sizes to which models are
fit.

\centerline{\bf 7.2 Point Spread Functions and Accuracy}

While an elliptical Gaussian was suggested as a possible
model for a model for a stellar point spread function, the
surface brightness profiles of stellar images tend to look
more like power laws.  The actual model used for stellar
images, consists of similar ellipses of the form
$$I(x,y) = I_0 \left( 1 + z^2 + {1 \over 2} \beta_4 (z^2)^2 + {1
\over 6} \beta_6 (z^2)^3 \right)^{-1} + I_s.  \nexteq$$ 
where 
$$ z^2
= {\left[{{1}\over{2}} \left({{x^2}\over{\sigma_x^2}} +
2\sigma_{xy}xy + {{y^2}\over{\sigma_y^2}}\right)\right]},
\nexteq$$ 
and 
$$x = (x'-x_0) \quad ; \quad y = (y'-y_0), \nexteq$$
with the nominal center of the image at $(x_0, y_0)$.  The
dimensionless quantities
$\beta_4$  and $\beta_6$ are user specifiable, but are
ordinarily taken to be unity.
There
are seven free parameters in this function: the shape
parameters $\sigma_x$, $\sigma_y$, and $\sigma_{xy}$; the
image center, $(x_0,y_0)$; the central intensity, $I_0$; and
the background intensity, $I_s$.

The choice of an analytic, as opposed to a tabulated
empirical point spread function was driven by several
considerations.  First was expedience.  The authors had
experience in fitting analytic functions but none in fitting
tabulated functions.  Next was speed of calculation.
Another was the hunch that one could do good {\it relative}
photometry within an image even with an imperfect
approximation to the point spread function, since the
systematic error made by adopting arising from the
approximate nature of the model should be the same for all
stars.  This is not strictly correct, since the profiles of
bright stars are weighted by the photon statistics of the
star itself, while the profiles of faint stars are weighted
by the photon statistics of the sky.  Results from a limited
set of experiments would indicate that the associated scale
errors are less than 0.01 magnitude per magnitude.

The reader will note that the free parameters in the above
equations are not the same as those in the output parameter
list.  The transformation to magnitudes, FWHM, axis ratio
and position, the external representation of the parameters,
is made only during when writing this list to disk.  For all
other purposes, DoPHOT uses as its internal representation
of the parameters the quantities indicated in equations (1)
- (3).  If desired, output files can easily be generated that
preserve this internal DoPHOT parameter representation (see
Appendix A).  Moreover, error files which contain the square 
roots of the diagonal covariance matrix elements for these 
variables can be generated as well (see
Appendix A).

\centerline{\bf 7.3 Phantom Stars and the Noise Array}

The most obvious shortcomings of adopting an analytic,
as opposed to empirical, point spread function is
the large residuals one gets from the best fitting model.
Since DoPHOT makes multiple passes through the object-subtracted
image, there is a danger that the program will trigger
on a positive residual and identify a spurious ``phantom''
star.  Phantom stars may be checked for in the ``synthetic''
image, generated by subtracting the object-subtracted
image from the original image.  Another useful diagnostic
is to plot the $(x,y)$ positions of objects using a plotting
program, keeping an eye out for unusual grouping of faint
objects around bright objects or at the edges of obliteration boxes.

DoPHOT uses its noise array to avoid such phantom stars.
Whenever an object is subtracted from an image, a user
specified fraction of the subtracted signal is added, in
quadrature, to the noise array.  The larger one makes this
fraction, the less likely one is to detect a spurious
object.  In increasing this fraction, one pays two
penalties: one is less likely to detect faint stars in the
vicinity of bright stars and the fits to objects which are
detected in the vicinity of another object will be more
uncertain than they would otherwise be.

While this scheme seems to work reasonably well at the
centers of images, phantom stars still appear on the
peripheries of bright stars when the actual stellar profile
has broader wings than the analytic model.  Ideally, one
ought to try an different analytic approximation.  Short of
that, one can adjust a factor which expands the shape
parameters used in augmenting the noise array by a user
specifiable factor, or, if the stellar objects are well-sampled, rebin
the data to achieve object profiles with FWHM of about 3 pixels.

\centerline{\bf 7.4 Star/Galaxy/Double-Star Classification}

When a subraster of pixels gives a significant signal when
tested against the stellar model filter, one doesn't know
whether one is seeing the light from a single star, from a
galaxy, or from two or more stars which lie relatively close
to each other.  

DoPHOT attempts to deal with this classification question
economically, taking advantage of the fact that it must
determine the shapes of objects to determine the parameters
associated with a ``typical'' star.  When those shape parameters
are significantly different from the typical image (by a
user specifiable amount) and when the sense of that
difference is that the area of the associated footprint is
larger, DoPHOT declares the object to be ``very big.''

The significance of the difference between the shape
parameters for an individual object and those for a
``typical'' star depend first, upon the errors in the fit
for the individual star and second, on the star-to-star
scatter in those parameters.  Given a perfect instrument,
and an ensemble of stars of the same brightness, one would
expect these to be the same.  For many instruments ({\it e.g.}
wiggly CCDs in fast beams) they are not.  DoPHOT computes a
star-to-star scatter in the stellar parameters that weights
the bright stars, those for which the accidental errors from
the fit are likely to be small, more heavily.  The scatter
for each shape parameter is added in quadrature to the
uncertainties for that parameter obtained from the fit to an
individual star.  The significance of the difference between
the measured and expected shape parameters is computed
taking this as the expected difference.

An {\it ad hoc} (and not at all satisfying) modification of the
above scheme is that the user is allowed to specify a
minimum scatter expected in the shape parameters.  This
keeps stars on the second pass through the data from being
classified as galaxies when the first few stars have an
unusually small scatter.

When an object is found to be ``big,'' DoPHOT attempts to
fit two typical stellar profiles to the subraster.  The
goodness of fit parameters return from the fits to the
single object and the two typical objects are compared, and
based on a user specifiable parameter a decision is made on
its classification.  If the object is double, it is
``split'' and two entries are made in the object list.
These entries are then subject to further testing and
splitting on subsequent passes through the image.  Galaxies
are object type 2, while siblings of split stars are object
type 3.  Objects of type 3 can be reclassified as galaxies
but never as single stars of type 1.

At high galactic latitudes, in regions of low star
densities, the user will want to adjust the parameter
which control the galaxy/double-star decision to
favor galaxies.   At low galactic latitudes and
in star clusters the decision should favor double-stars.
The synthetic image is useful for diagnosing how this
parameter should be set.

One difficulty with modeling objects is that at
progressively lower signal-to-noise ratios one can support
fewer and fewer free parameters.  Since the fitting in
DoPHOT is carried out iteratively, the symptom of too low a
signal-to-ratio is a failure to converge.  There is a user
specifiable signal-to-noise limit such that only objects
with higher signal to noise are fitted for shape parameters.
Objects which are fainter than this are classified as object
type 7 in the output list and are only fit for sky, $(x,y)$ position,
and central intensity while adopting the current best estimates of
the shape parameters as fixed.

At high galactic latitudes some large fraction of such faint
objects may nonetheless be galaxies.  To provide at least
some handle on whether or not such objects are
galaxies/double-stars or single stars, an ``extendedness''
parameter is computed for every object.  Using the typical
stellar parameters for an initial guess, the fitting program
is allowed to determine the first step toward improving the
shape parameters.  The change in the goodness-of-fit parameter
predicted by its second derivative matrix is then estimated.
If the sense of the change is to decrease the area of the
footprint, this change is reported as a negative number;
otherwise it is reported as positive.  

{\it Limited} tests indicate that while this quantity does
help to separate stars from galaxies (and from globular
clusters in nearby galaxies) its value is not independent
of the brightness of the object.  The user is encouraged
to investigate this further, perhaps by comparing images of a given
field obtained in very different seeing conditions and for which the
user can readily classify objects in the better-seeing data that are
harder to classify in the poorer data.

\centerline{\bf 7.5 Completeness and Faint-End-Bias}

The model that is used as the filter for measuring the
signal-to-noise ratio for a potential object is identical to
that of the typical stellar profile used for fitting the
object.  It should not come as a surprise, therefore, that
when the numbers of objects identified is plotted as a
function of apparent magnitude obtained from the fit, one finds
that it drops to zero quite abruptly.  The detection and the
measurement are carried out consistently.

The naive user might be deceived into thinking that the
sample of detected is remarkably complete.  It isn't.
Consider several stars for which the expected
signal-to-noise ratio is exactly the limiting value.  Some
of those will be superposed on positive noise fluctuations,
and will be detected, with measured fluxes brighter than
their true fluxes.  Others will be superposed on negative
noise fluctuations, and missed.  Not only will the sample be
incomplete, the estimated magnitudes will be biassed by an
amount which depends upon the logarithmic slope of their
number-magnitude distribution and upon the limiting
signal-to-noise ratio.

\centerline{\bf 7.6 Model Sky and Average Sky}

For every star in the object list, DoPHOT produces an
estimate of the sky brightness.  It is therefore
possible to construct a ``model'' of the variation
of the sky brightness over the face of the chip.
This model is then used to {\it estimate} the sky value
when a potential object is filtered through the
typical stellar profile.

If the object passes through the filter using
the model sky, it is tested once again, this time
using a weighted average of the sky computed from
the fit subraster.  This second test avoids spurious
detections where the background level is changing
in a way which is not modeled.

The model for the sky need not be a constant.  Typically
it is a plane, which allows for a small gradient across
the field.  An option which has proved quite useful
is modeling the sky as a ``Hubble'' profile plus a
constant.   In fitting this model (and the much simpler
plane) the data points are the values of the sky derived
from fits to individual stars.  The ``Hubble'' profile
requires a relatively large number of points (of order 50)
to converge.

\centerline{\bf 7.7 Warmstarts}

There are a variety of circumstances under which it is
helpful to give DoPHOT a starting list of objects.  For
example, having done a first pass on the data, one might
want to try lowering the threshold without having to start
over from the beginning.  Or might want to insert
an object by hand into the list, or to change the parameter
which controls whether objects are stars or galaxies.
In the latter case, if one thought that DoPHOT had erred
on the side of classifying galaxies as double-stars, one
would need to remove the offending type 3 objects before
re-starting the program.

In galaxy fields, we have found it useful to `train' DoPHOT
on preselected stars so it doesn't misclassify bright, narrow 
galaxies as stars before it reaches the dimmer stellar objects.

\centerline{\bf 7.8 Obliteration}

Bright stars present a several problems for automated
photometry.  Since detectors often go non-linear and then
saturate at high surface brightness levels, it is difficult
to subtract badly overexposed stars from the image.  Rather
than try to fit and subtract such an object, DoPHOT allows
for the automatic the excision of a rectangular subraster
around bright objects.

The user has considerable (perhaps too much) control over
this process.  There is a user-adjustable parameter which
places a limit on the allowable number of saturated pixels.
Objects with more than this number are excised from the
image, ``obliterated.''  There is also a user specified
upper limit to the central intensity of an object.  If the
fit to a subraster gives a larger value than this, the star
is excised.

The size of the region excised will depends upon the best
available estimate of the central intensity of the object.
If this has not been fit for, it is taken to be proportional
to the number of saturated pixels, with the constant of
proportionality specified by the user.  The surface
brightness level down to which pixels are excised is also
user specifiable.

In the event of exceedingly bright objects, or diffraction
spikes from such objects, the user can specify a region to
be excised by entering an object of type 8 in the input
object list and ``warmstarting'' the program.  
The test image provides an example of how the user
may explicitly obliterate regions of an image.  The user may 
choose to excise rectangular regions of the image or elliptical
ones. 

\centerline{\bf 7.9 Trouble}

A number of circumstances arise where, try as it might,
DoPHOT cannot come up with magnitudes or shapes for objects.
The most common of these is when there are too few samples
to do an adequate job of fitting the model.  There are two
user adjustable parameters which specify the fraction of the
fit subraster that must be present for an object to be fit.
If too few pixels are present for the shape of an object to
be determined, it is classified as object type 5.  If an
attempt to determine the shape of an object fails to converge,
it is classified as object type 9.

If an attempt to determine the magnitude of an object using
the typical shape fails to converge, the object is
classified as type 4.  If too few pixels are present for the
object to be fit using the typical shape parameter, the
object is declared object type 6.  Objects of this type are
{\it not} subtracted from the image -- they are said to have
been ``deactivated.''  They are left in the object list so
that a record is left that something out of the ordinary has
happened.  If the signal-to-noise ratio of an object which
was previously identified is found to be less than the user
specified value for the identification of objects, the object
is likewise classified as type 6, and ``deactivated.''  

\centerline{\bf 7.10  Internal Parameter Representation}

DoPHOT fits for the parameters ${\sigma_x}^2$ and
${\sigma_y}^2$ rather than $\sigma_x$ and $\sigma_y$ so that
the program can recover more easily from iterations which
might otherwise make $\sigma_x$ and $\sigma_y$ negative.
The quantity $\sigma_{xy}$ is measured in {\it inverse}
square pixels so that round images produces parameters which
behave well.  When fitting models, DoPHOT divides the
central intensity and sky values by 100 to avoid overflows.

When fitting analytic objects, DoPHOT
treats the logarithm of the central intensity as the free
parameter; this allows for more robust convergence of single 
stars and galaxies and recovery from overzealous attempts to
decrease the brightness of one of the two elements of a
double star for double objects.  







































