\centerline{\bf 5.  IMAGE FORMATS}

All image reading and writing in DoPHOT 4.0 is handled 
by CFITSIO, and the only accepted file types are FITS format images.  
CFITSIO can handle almost all standard types of data, but DoPHOT 
4.0 will process the pixel data as 32 bit integers regardless.  DoPHOT 
veterans may recall that versions 3.0 and earlier only accepted 16 bit 
signed integer data and processed it as the same -- this has been 
modified in DoPHOT 4.0.  As DoPHOT 4.0 processes data as 32 bit 
integers, it outputs FITS files of the same type.  DoPHOT will preserve all
header information from input files in the output files if the input
files have 32 bits per pixel (BITPIX = 32 or -32).  Barring this matching
file size for input and output files, a minimal header is created for the 
output files.

